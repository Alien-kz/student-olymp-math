\documentclass[12pt, a5paper]{article}
\usepackage[utf8]{inputenc}
\usepackage[T2A]{fontenc}
\usepackage[russian]{babel}
\usepackage{amsmath}
\usepackage{amsfonts}
\usepackage{amssymb}
\usepackage[left=2cm,right=1cm,top=2cm,bottom=2cm, twoside]{geometry}
\usepackage{lscape}
\usepackage{comment}


\usepackage{pgf,tikz}
\usetikzlibrary{arrows}

\binoppenalty=10000
\relpenalty=10000	

\newcommand{\header}[2]
{
	\subsection{{#1}}
	\begin{center}
%	\textbf{#1}\\
	#2
	\end{center}
}

\sloppy

%space from foothead
\headsep=12pt
\setlength{\headheight}{28pt} 

%header
\usepackage{fancyhdr}
\fancypagestyle{mystyle}
{
	\fancyhead{}
	\fancyhead[LE]{\nouppercase\leftmark}		% LE -> Left part on Even pages
	\fancyhead[RO]{\nouppercase\rightmark}		% RO -> Right part on Odd pages
}

%center sections name
\renewcommand\thesection{}
\renewcommand\thesubsection{}
\usepackage[explicit]{titlesec}
\titleformat{\section}{\normalfont\bfseries\center}{}{1em}{#1}
\titleformat{\subsection}{\normalfont\bfseries\center}{}{1em}{#1}

\begin{document}

\begin{titlepage}
\begin{center}
\vfill

Московский государственный университет\\
имени М.В.~Ломоносова\\
Казахстанский филиал\\

\vfill

{\large\bf 
Студенческие олимпиады по математике\\
Казахстанского филиала МГУ.\\
Задачи и указания. \\
2008--2018 гг.\\}

\vfill

Астана\\
2018
\end{center}
\end{titlepage}

\setcounter{page}{2}


\thispagestyle{empty}

\noindent{\bf УДК} \\
{\bf ББК}\\
{\bf Л}
\vspace{0.7 cm}


{\bf ISBN}  

\vspace{0.7 cm}

В настоящем сборнике представлены более 100 задач студенческих олимпиад по математике Казахстанского филиала МГУ имени М.В.Ломоносова за период с 2008 по 2018 год.

Сборник  адресован  всем интересующимся олимпиадным движением.

\vspace{0.5 cm}

{\bf Составители:}

Абдикалыков А.К., Баев А.Ж., Васильев А.Н.

\vspace{0.5 cm}


{\bf TeX--верстка:}

Баев А.Ж.

\newpage
\begin{small}
\begin{center}
\textbf{Предисловие}
\end{center}
В Казахстанском филиале  МГУ имени М.В.~Ломоносова ежегодно проводятся студенческие олимпиады по математике. На базе Казахстанского филиала проводится также Республиканский этап студенческой предметной олимпиады по математике (2016 и 2017~гг.). Проведение олимпиады в Филиале (уже во второй раз) является результатом побед наших студентов на данной олимпиаде в предыдущие годы.

Ежегодно в декабре студенты Казахстанского филиала проходят отбор внутри университета для участия на Республиканском этапе. В связи со спецификой обучения в Филиале, на олимпиаде участвуют студенты первого и второго курсов механико--математического факультета и факультета вычислительной математики и кибернетики (студенты третьего и четвертого курса продолжают обучение в Москве). На олимпиаде обычно предлагается от 7 до 10 заданий по математическому анализу, алгебре, геометрии, теории чисел, дискретной математике и по крайней мере одна задача из коллекции школьных олимпиад по математике. Такой вариант позволяет первокурсникам свободно конкурировать со студентами второго курса. Победители справляются, как правило, с 4 или 5 заданиями.

В данном сборнике представлены тексты около 100 задач олимпиад последних десяти лет, к которым даны указания, составленные преподавателями Казахстанского филиала. При подготовке олимпиад использованы материалы из различных сборников студенческих и школьных олимпиад, а также авторские задачи. Авторами задач являются преподаватели филиала Абдикалыков А.К., Баев А.Ж., Васильев А.Н., которые активно участвуют в олимпиадном движении Казахстана. 

Составители выражают благодарность профессору  Нурсултанову~Е.Д. и доценту Бекмаганбетову~К.А. за оказание содействия при подготовке сборника.

\begin{flushright}
Директор Казахстанского филиала

А.В.Сидорович
\end{flushright}
\end{small}

\newpage

\mbox{}

\newpage

\tableofcontents

\newpage

\pagestyle{mystyle}

\section{Условия задач олимпиад Казахстанского филиала}

\header{2008--2009}{7 декабря 2008}
\input{problem/2008}
\newpage

\header{2009--2010}{6 декабря 2009}
\input{problem/2009}
\newpage

\header{2010--2011}{12 декабря 2010}
\input{problem/2010}
\newpage

\header{2011--2012}{10 декабря 2011}
\input{problem/2011}
\newpage

\header{2012--2013}{21 декабря 2012}
\input{problem/2012}
\newpage

\header{2013--2014}{20 декабря 2013}
\input{problem/2013}
\newpage

\header{2013--2014 (дополнительный тур)}{15 марта 2014}
\input{problem/2014-bonus}
\newpage

\header{2014--2015}{10 декабря 2014}
\begin{center}
\begin{tabular}{|l|l|l|c|*{10}{p{0.3cm}|}c|c|}
\hline
№ & Участник & Спец & Курс & 1 & 2 & 3 & 4 & 5 & 6 & 7 & 8 & 9 & 10 & $\Sigma$ & Диплом\\
\hline
1 & Амир Мирас &  ВМК & 2 & 0 & 10 & 0 & 9 & 10 & 10 & 0 & 0 & 0 & 0 & 39 & 1\\
\hline
2 & Журавская Александра &  ВМК & 1 & 0 & 0 & 0 & 0 & 10 & 10 & 10 & 0 & 8 & 0 & 38 & 1\\
\hline
3 & Булгаков Анатолий &  ВМК & 2 & 2 & 2 & 0 & 0 & 10 & 9 & 0 & 0 & 0 & 0 & 23 & 3\\
\hline
4 & Шокетаева Надира &  ММ & 2 & 3 & 0 & 0 & 0 & 5 & 10 & 0 & 0 & 0 & 0 & 18 & 3\\
\hline
5 & Абайулы Ерулан &  ВМК & 1 & 0 & 0 & 0 & 0 & 10 & 7 & 0 & 0 & 0 & 0 & 17 & 3\\
\hline
6-8 & Таскынов Ануар &  ВМК & 2 & 0 & 0 & 0 & 0 & 0 & 10 & 0 & 0 & 0 & 0 & 10 & \\
\hline
6-8 & Токтаганов Адильхан &  ММ & 1 & 0 & 0 & 0 & 0 & 0 & 10 & 0 & 0 & 0 & 0 & 10 & \\
\hline
6-8 & Таранов Денис &  ВМ & 2 & 0 & 0 & 0 & 0 & 9 & 0 & 0 & 1 & 0 & 0 & 10 & \\
\hline
9 & Батырбеков Аскар &  ММ & 1 & 0 & 0 & 0 & 0 & 0 & 9 & 0 & 0 & 0 & 0 & 9 & \\
\hline
10 & Кенесова Аида &  ВМ & 1 & 0 & 0 & 0 & 0 & 0 & 2 & 0 & 0 & 0 & 0 & 2 & \\
\hline
11 & Даку Ансар &  ВМ & 1 & 0 & 0 & 0 & 0 & 1 & 0 & 0 & 0 & 0 & 0 & 1 & \\
\hline
\end{tabular}

\end{center}
\newpage

\header{2015--2016}{19 декабря 2015}
\input{problem/2015}
\newpage

\header{2016--2017}{10 декабря 2016}
\begin{flushright}
Стоимость задач: \\
10 баллов каждая задача.\\
\end{flushright}

\begin{enumerate}
\item Введём функцию
$$
f(n) = \bigl[\sqrt 1\,\bigr] + \bigl[\sqrt 2\,\bigr] + \bigl[\sqrt 3\,\bigr] + \hdots + \bigl[\sqrt{n^2-1}\,\bigr] + \bigl[\sqrt{n^2}\,\bigr],
$$
где $[x]$ --- наибольшее целое число, не превышающее $x$. Опишите функцию, которая вычисляет $f(n)$ для данного натурального $n$, не используя при этом операцию извлечения корня и вещественную арифметику.

\item На декартовой координатной плоскости нарисованы две полупараболы: график функции $y = x^2$ $(x \geqslant 0)$ и его копия, повёрнутая на прямой угол по часовой стрелке. Эти две кривые отсекают от прямой, параллельной оси ординат, отрезок длины $L$. Обозначим через $S(L)$ --- площадь отсечённой фигуры.

a) Докажите, что $S(L) > 1$ при $L > 2$;

б) Напишите функцию, которая вычисляет $S(L)$ для данного положительного вещественного числа $L$.



\item Найдите все дифференцируемые функции $f\colon \mathbb R\rightarrow\mathbb R$, удовлетворяющие соотношению
$$
f(x-y) + f(x+y) = f'(x^2 + y^2)
$$
для любых $x,y\in\mathbb R$.



\item  Дана функция $f\colon [0, 2n]\rightarrow\mathbb R$. Пусть $f_i = f(i)$ --- значения функции во всех целых $i$ от 0 до $2n$. Дана переменная $S$ вещественного типа с начальным значением 0. За один ход робот может выбрать целое $i$ от 1 до $2n-1$, затем добавить к переменной $S$ или вычесть из нее среднее арифметическое значений функции $f(x)$ в узлах~$i-1$,~$i$,~$i+1$:
$$S := S \pm \frac{f_{i-1}+f_i+f_{i+1}}{3}.$$ 
Может ли робот за конечное число ходов получить в переменной $S$ значение
$$I = \frac{1}{3} \left(f_0 + 4\sum\limits_{k=1}^{n}{f_{2k-1}} + 2\sum\limits_{k=1}^{n-1}{f_{2k}} + f_{2n}\right),$$
которое является приближением интеграла $\displaystyle \int\limits_0^{2n} f(x)\,dx$, если\\
а) $f(0) = f(2n) = 0$;\\
б) $f(0) \ne 0$, $f(2n) \ne 0$?

\item  Дана некоторая условная машина, состоящая из памяти в $n$ бит и указателя, который в каждый отдельный момент находится над какой-то из этих n ячеек. Перед запуском программы в память записывается некоторое натуральное число $m$ в двоичной системе счисления, а указатель устанавливается над крайним правым (младшим) битом числа. Язык программирования для этой машины состоит из следующих команд:

\begin{center}
\begin{tabular}{|c|c|p{14cm}|}
\hline
{\bf L} & left & сместить указатель налево на одну ячейку, если это возможно, иначе завершить программу\\
\hline
{\bf R} & right & сместить указатель направо на одну ячейку, если это возможно, иначе завершить программу\\
\hline
{\bf C} & change & изменить значение бита в текущей ячейке на противоположное\\
\hline
{\bf A} & again & перейти к выполнению первой команды\\
\hline
{\bf S} & skip & пропустить две следующие команды, если в текущей ячейке 0\\
\hline
{\bf F} & finish & завершить выполнение программы\\
\hline
\end{tabular}
\end{center}

Команды записываются в одну строку и выполняются в последовательном порядке, слева направо. При этом запись программы обязана оканчиваться командой \textbf{A} или \textbf{F}. Напишите для этой абстрактной машины следующие программы:

а) заменить данное число на $(m - 1)$;

б) заменить данное число на $(2^n - m - 1)$;

в) изменить на противоположный его старший (крайний слева) бит.

\textit{Примеры:}

а) программа, обнуляющая все ячейки: \textbf{SSCLA};

б) программа, которая изменяет второй справа бит, если крайний справа бит нулевой: \textbf{SFFLCF}.


\item Из квадратной однородной пластины со стороной 1 случайным образом вырезается квадрат со сторонами, равными $2a$ и параллельными сторонам исходного квадрата. При этом центр квадрата --- это случайная величина, равномерно распределённая по всем допустимым положениям (квадрат со стороной $(1 - 2a)$).

a) Найдите вероятность $p(a)$ того, что центр тяжести полученной фигуры лежит в вырезанной области.

б) Опишите функцию $p(a)$, которая вычисляет указанную вероятность приблизительно, считая при этом, что нам не известен метод нахождения центра тяжести произвольной фигуры, однако мы можем найти центр тяжести конечного множества точек одинаковой массы.

\end{enumerate}

\newpage

\header{2017--2018}{9 декабря 2017}
\begin{center}
\begin{tabular}{|l|l|l|c|c|c|c|c|c|c|c|c|}
\hline
№ & Участник & ВУЗ & Курс & 1 & 2 & 3 & 4 & 5 & 6 & $\Sigma$ & Диплом \\
\hline
1 & Жанбырбаев Есеналы & КБТУ & 2 & 10 & 10 & 10 & 3 & 0 & 10 & 43 & 1 степени \\
\hline
2 & Бекмаганбетов Бекарыс & КФ МГУ & 1 & 10 & 10 & 10 & 2 & 10 & 0 & 42 & 2 степени \\
\hline
3 & Сайланбаев Алибек & НУ & 4 & 10 & 10 & 10 & 2 & 0 & 8 & 40 & 2 степени \\
\hline
4 & Аманкелды Акежан & НУ & 4 & 9 & 9 & 10 & 5 & 0 & 0 & 33 & 3 степени \\
\hline
5 & Жанахметов Султан & НУ & 3 & 10 & 10 & 10 & 2 & 0 & 0 & 32 & 3 степени \\
\hline
6 & Шакиев Александр & МУИТ & 2 & 10 & 9 & 10 & 0 & 0 & 0 & 29 & 3 степени \\
\hline
\end{tabular}
\end{center}
\newpage

\header{2017--2018 (дополнительный тур)}{13 марта 2018}
\input{problem/2018-bonus}
\newpage

\header{Республиканская олимпиада по математике 2016}{01 апреля 2016}
\input{problem/2016-rep-math}
\newpage

\header{Республиканская олимпиада по МКМ 2016}{01 апреля 2016}
\input{problem/2016-rep-mcm}
\newpage

\header{Республиканская олимпиада по математике 2017}{13 апреля 2017}
\input{problem/2017-rep-math}
\newpage

\mbox{}

\newpage

\section{Указания}

\header{2008--2009}{7 декабря 2008}
\input{solution/2008-sol}
\newpage

\header{2009--2010}{6 декабря 2009}
\input{solution/2009-sol}
\newpage

\header{2010--2011}{12 декабря 2010}
\input{solution/2010-sol}
\newpage

\header{2011--2012}{10 декабря 2011}
\input{solution/2011-sol}
\newpage

\header{2012--2013}{21 декабря 2012}
\input{solution/2012-sol}
\newpage

\header{2013--2014}{20 декабря 2013}
\input{solution/2013-sol}
\newpage

\header{2013--2014 (дополнительный тур)}{15 марта 2014}
\input{solution/2014-bonus-sol}
\newpage

\header{2014--2015}{10 декабря 2014}
\input{solution/2014-sol}
\newpage

\header{2015--2016}{19 декабря 2015}
\input{solution/2015-sol}
\newpage

\header{2016--2017}{10 декабря 2016}
\input{solution/2016-sol}
\newpage

\header{2017--2018}{9 декабря 2017}
\input{solution/2017-sol}
\newpage

\header{2017--2018 (дополнительный тур)}{13 марта 2018}
\input{solution/2018-bonus-sol}
\newpage
%%%%%%%%%%%%%%%%%%%%

\header{Республиканская олимпиада по математике 2016}{01 апреля 2016}
\begin{enumerate}

\item (Васильев А.Н.)

Заметим, что справедливо разложение $$x^2-y^2+2x+2y = (x+y)(x-y+2).$$ Поэтому натуральное число представимо в этом виде тогда, и только тогда, когда раскладывается на произведение двух множителей одной четности. Ясно, что это все числа, которые дают остаток отличный от 2 при делении на 4.

Пример для нечетного $n$: $x = \frac{n-1}{2}$, $y = \frac{3-n}{2}$.

Пример для $n$, кратного 4: $x = y = \frac{n}{4}$.
 
\item (Васильев А.Н.) 

а) Легко понять, что функция кусочно--постоянная. Причем количество промежутков постоянства конечно и равно 10. Значит, функция интегрируема по Риману. 
\\
б) Найдем промежуток, на котором первая цифра числа $2^x$ равна $k$:
$$1 + \frac{k}{10} \le 2^x < 1 + \frac{k+1}{10},$$
$$\log_2 \left( 1 + \frac{k}{10} \right) \le x < \log_2 \left( 1 + \frac{k+1}{10} \right).$$

Тогда наш интеграл можно записать в виде суммы:
\begin{multline*}
\int\limits_{0}^{1} \alpha(x) dx = \\
= \sum_{k=0}^{9} k \left( \log_2 \left( 1 + \frac{k+1}{10} \right) - \log_2 \left( 1 + \frac{k}{10} \right) \right) = \\
= \sum_{k=0}^{9} k \left( \log_2 (k+11) - \log_2 (k+10) \right) = \\
= \sum_{k=0}^{9} k \log_2 (k+11) -  \sum_{k=0}^{8} (k + 1) \log_2 (k + 11) = \\
= 9 \log_2 20 - \sum_{k=0}^{8} \log_2 (k + 11) = \log_2 \frac{20^9}{11 \cdot 12 \cdot \ldots \cdot 19}
\end{multline*}

Требуется доказать, что
$$2^7 < \left( \frac{20^9}{11 \cdot 12 \cdot \dots \cdot 19} \right)^2 < 2^9.$$

Докажем левую часть неравенства. Заметим, что по неравенству Коши 
$$(10+k) * (20-k) < \left( \frac{10+k+20-k}{2} \right)^2 = 15^2.$$
Отсюда получается оценка слева:
$$\left( \frac{20^9}{11 \cdot 12 \cdot \ldots \cdot 19} \right)^2 > \left( \frac{20^9}{15^9} \right)^2 = \frac{2^{36}}{3^{18}}.$$

Остается доказать, что $2^{29} > 3^{18}$. Заметим, что $2^8>3^5$ и $2^5>3^3$. Перемножив три раза первое неравенство и один раз второе, получим требуемое.

Докажем правую часть неравенства. Заметим, что верно следующее неравенство:
$$(10+k) (20-k) = 200 + k (10 - k) > 200.$$

Значит, оценку справа можно получить так:
$$\left( \frac{20^9}{11 \cdot 12 \cdot \ldots \cdot 19} \right)^2 < \left( \frac{400^4 \cdot 20}{200^4 \cdot 15} \right)^2 = 2^8 \left( \frac{4}{3} \right)^2 < 2^9.$$


\item (Фольклор) Первое решение (<<наивное>>). Можно доказать более общее утверждение:

\textit{В любом конечном поле $F \ne Z_2$ сумма всех элементов равна нулю.}

Пусть $F$ --- конечное поле и $a_1$, $a_2$, ..., $a_n$ --- все его элементы. Если $F \ne Z_2$, то существует элемент $a$, отличный от нуля и единицы. Тогда $a a_1$, $a a_2$, ..., $a a_n$ попарно различны, следовательно
$$F = \{ a_1, ..., a_n \} = \{ a a_1, ..., a a_n \} .$$
Отсюда $S = \sum_{i=1}^{n} a_i = \sum_{i=1}^n a a_i = a S$, откуда следует, что $S = 0$.

Второе решение(существенно использующее структуру конечного поля). Утверждение из предыдущего решения можно доказать и по-другому. Ненулевые элементы поля образуют группу по умножению, а порядок элемента группы делит порядок группы (по теореме  Лагранжа). Следовательно, любой элемент поля $F$ является корнем многочлена $x^n - x = 0$, где $n$ --- количество элементов поля. С другой стороны, по другой теореме Лагранжа, у этого многочлена не более $n$ корней. Иными словами, указанный многочлен имеет своими корнями все элементы поля. Применяя теорему Виета, получаем требуемое. 

Третье решение (еще одно). У каждого ненулевого элемента $x$ есть обратный $x^{-1}$, причем $x \ne x^{-1}$ при $x \ne \pm 1$. Следовательно, все ненулевые элементы, кроме $\pm 1$, разбиваются на пары с произведением 1. Поэтому произведение всех элементов поля равно $-1$. Из условия задачи следует, что $-1 \ne 1$. Следовательно, характеристика поля отлична от 2. Тогда любой ненулевой элемент отличается от своего противоположного, то есть все ненулевые элементы разбиваются на пары с нулевой суммой. Что означает, что сумма всех ненулевых элементов поля равна нулю. Добавление нуля сумму не изменяет. Утверждение доказано.    

\item  (Баев А.Ж.) 

Факт 1 (оптическое свойство эллипса): луч, направленный из одного фокуса после отражения от внутренней стороны эллипса проходит через другой фокус. То есть $\angle(F_1M, SM) = \angle(SM, F_2M)$, где $\angle(l_1, l_2)$ обозначает ориентированный угол между прямыми. Как следствие, получаем, что $\angle F_1MS + \angle F_2MS = \pi$. По условию, $\angle F_2MS = \angle DMS$. Откуда получаем, что $F_1$, $M$, $D$ лежат на одной прямой. Аналогично, $F_2$, $N$, $E$ лежат на одной прямой.

Факт 2 (определение эллипса). Сумма расстояний от фокусов до точек на эллипсе постоянна. Как следствие $F_1M + MF_2 = F_1N + NF_2$. Так как треугольники $F_2MS$ и $DMS$ симметричны относительно прямой $MS$, то и треугольники $F_1MF_2$ и $AMD$ тоже симметричны и, соответственно, равны. Аналогично, симметричны и равны треугольники $F_1NF_2$ и $BNE$.

\begin{center}
\includegraphics[width=8cm]{pictures/2016-republic}
\end{center}

а) 
$$AF_2 = AM + MF_2 = F_1M + MF_2 =$$
$$= F_1N + NF_2 = BN + NF_2 = BF_2.$$ 
Значит, $CF_2$ --- медиана треугольника $ABC$.

б) Четырехугольник $F_1DCE$ вписан в окружность, так как $\angle F_1DC + \angle F_1EC = \angle MF_2S + \angle NF_2S = \pi$. Так как в этом четырехугольнике две смежные стороны равны ($F_1D = F_1E$), то $CF_1$ --- биссектриса треугольника $ABC$.

\item (Клячко А.А.)

1) Рассмотрим случай четного $n$. Тогда каждый из игроков полностью контролирует $\frac{n}{2}$ столбцов (при этом не имеет значения, кто делает первый ход). Ясно, что Максималист может сделать свои столбцы линейно независимыми и обеспечить ранг матрицы минимум $\frac{n}{2}$. Также ясно, что Минималист может сделать все свои столбцы нулевыми, ограничив ранг матрицы $\frac{n}{2}$.

Ответ для четного $n$: $\frac{n}{2}$.

2) Пусть $n$ нечетно. Тогда, если мы раскрасим клетки таблицы в черный и белый цвета в шахматном порядке, каждый из игроков будет контролировать клетки одного цвета. 

а) Пусть Максималист делает первый ход. Тогда он сможет сделать ранг матрицы максимальным, то есть равным $n$. Опишем его стратегию. Она состоит в том, что, заполняя очередную диагональную клетку, он следит за тем, чтобы соответствующий главный (угловой) минор был отличен от нуля. Это всегда можно обеспечить, поскольку этот минор разлагается по своей последней строке, а алгебраическое дополнение последнего элемента не равно нулю. 
Ответ для нечетного $n$, когда Максималист делает первый ход: $n$. 

б) Пусть Минималист делает первый ход. Тогда он сможет обеспечить равенство нулю определителя всей матрицы: заполняя очередную диагональную клетку (кроме последней), он следит за тем, чтобы соответствующий угловой минор был отличен от нуля, а в конце обнуляет определитель всей матрицы. Значит, он сможет гарантировать ранг меньше $n$. С другой стороны, Максималист сможет обеспечить, чтобы минор, полученный вычеркиванием последней строки и первого столбца, был отличен от нуля (аналогично пункту 2 а)). Тем самым, ранг матрицы будет равен по крайней мере $n-1$. 

Ответ для нечетного $n$, когда Минималист делает первый ход: $n-1$. 
 
\item (Баев А.Ж.)

1 шаг. Подставим в соотношение $x = \frac{t}{t-1}$, где $t > 1$. Получим
$$f'\left(\frac{t}{t-1}\right) = f(t) + f\left( \frac{t}{t-1} \right) .$$

Получим свойство:
$$f' \left(\frac{t}{t-1} \right) = f'(t).$$

2 шаг. Продифференцируем исходное соотношение по $x$.
$$ f''(x) = - \frac{1}{(x-1)^2} f' \left( \frac{x}{x-1} \right) + f'(x) .$$

После замены из свойства, получаем:
$$ \frac{f''(x)}{f'(x)} = 1 - \frac{1}{(x-1)^2} .$$

Уравнение интегрируется по частям:
$$ f'(x) = C e^{x + \frac{1}{x-1}}.$$

Добавим условие на бесконечности и найдем $C = 1$:
$$ f'(x) = 2 e^{x + \frac{1}{x-1}}.$$

3 шаг. Заметим, что если в исходное дифференциальное уравнение мы подставим $x = 2$, то получим $f(2) = \frac{1}{2} f'(2)$. Значит:
$$ f(2) = e^3.$$

Осталось доказать, что $e^3 < 20.16$. Заметим, что для проверки этого неравенства грубых оценок типа $e<3$ или $e<2.8$ недостаточно, требуется более точная: $e<2.72$.

\end{enumerate}
\newpage

\header{Республиканская олимпиада по МКМ 2016}{01 апреля 2016}
\input{solution/2016-rep-mcm-sol}
\newpage

\header{Республиканская олимпиада по математике 2017}{13 апреля 2017}
\input{solution/2017-rep-math-sol}
\newpage

%%%%%%%%%%%%%%%%%%%%%%%%%%%%%%%%%%%%%%%%%%%%

\begin{landscape}

\section{Результаты}

\begin{scriptsize}


\header{2012--2013}{21 декабря 2012}
\input{result/2012-res}
\newpage

\header{2013--2014}{20 декабря 2013}
\input{result/2013-res}
\newpage

\header{2013--2014 (дополнительный тур)}{15 марта 2014}
\input{result/2014-bonus-res}
\newpage

\header{2014--2015}{10 декабря 2014}
\input{result/2014-res}
\newpage

\header{2015--2016}{19 декабря 2015}
\input{result/2015-res}
\newpage

\header{2016--2017}{10 декабря 2016}
\input{result/2016-res}
\newpage

\header{2017--2018}{9 декабря 2017}
\input{result/2017-res}
\newpage

\header{2017--2018 (дополнительный тур)}{13 марта 2018}
\input{result/2018-bonus-res}
\newpage
%%%%%%%%%%%%%%%%%%%%

\header{Республиканская олимпиада по математике 2016}{01 апреля 2016}
\begin{tabular}{|l|l|l|l|c|*{6}{p{0.3cm}|}c|c|}
\hline
№ & Участник & ВУЗ & Город & Курс & 1 & 2 & 3 & 4 & 5 & 6 & $\Sigma$ & Диплом\\
\hline
1 & Журавская Александра & КФ МГУ  & Астана & 2 & 10 & 9 & 0 & 0 & 9 & 0 & 28 & 1\\ 
\hline
2 & Аманкелді Әкежан & НУ  & Астана &  2 & 3 & 9 & 2 & 0 & 1 & 7 & 22 & 2\\ 
\hline
3 & Турганбаев Сатбек & КФ МГУ  & Астана & 2 & 10 & 7 & 3 & 0 & 0 & 1 & 21 & 2\\ 
\hline
4 & Полищук Руслан & НУ  & Астана & 1 & 10 & 7 & 0 & 0 & 0 & 3 & 20 & 3\\ 
\hline
5 & Жанахметов Султан & НУ  & Астана & 2 & 10 & 5 & 0 & 0 & 1 & 3 & 19 & 3\\ 
\hline
6 & Токтаганов Адиль & КФ МГУ  & Астана & 2 & 8 & 0 & 0 & 10 & 1 & 0 & 19 & 3\\ 
\hline
7 & Сахит Аңыз & КазНУ  & Алматы & 2 & 4 & 5 & 4 & 0 & 2 & 0 & 15 & грамота\\ 
\hline
8 & Пикулина Алиса & КФ МГУ  & Астана & 1 & 10 & 0 & 2 & 0 & 0 & 0 & 12 & -\\ 
\hline
9 & Жакатаев Еркебулан & КазНУ  & Алматы & 2 & 6 & 3 & 3 & 0 & 0 & 0 & 12 & грамота\\ 
\hline
10 & Бейсембаев Жаслан & ПГУ  & Павлодар & 2 & 10 & 0 & 0 & 0 & 0 & 1 & 11 & грамота\\ 
\hline
11 & Нуртазин Руслан & КазНУ  & Алматы & 4 & 8 & 0 & 0 & 0 & 2 & 1 & 11 & грамота\\ 
\hline
12 & Сеилов Айтмухамед & КФ МГУ  & Астана & 1 & 4 & 0 & 4 & 0 & 2 & 0 & 10 & -\\ 
\hline
13 & Али Азамат & АРГУ  & Актобе & 1 & 10 & 0 & 0 & 0 & 0 & 0 & 10 & грамота\\ 
\hline
14 & Болотников Димитрий & КФ МГУ  & Астана & 1 & 4 & 0 & 0 & 0 & 5 & 0 & 9 & \\ 
\hline
15 & Умербекова Алина & НУ  & Астана & 3 & 4 & 1 & 0 & 0 & 3 & 0 & 8 & \\ 
\hline
№ & Участник & ВУЗ & Город & Курс & 1 & 2 & 3 & 4 & 5 & 6 & Итог & Диплом\\
\hline
16 & Коробов Павел & КФ МГУ  & Астана & 1 & 0 & 6 & 2 & 0 & 0 & 0 & 8 & \\ 
\hline
17 & Қайыров Қонысбек & АРГУ  & Актобе & 2 & 6 & 0 & 0 & 0 & 0 & 1 & 7 & \\ 
\hline
18 & Рахматуллаева Диера & ЮКГУ  & Шымкент & 2 & 4 & 0 & 0 & 0 & 0 & 1 & 5 & \\ 
\hline
19 & Тыныштықбай Абылай & КарГУ  & Караганда & 4 & 3 & 1 & 0 & 0 & 1 & 0 & 5 & \\ 
\hline
20 & Зекенгалиева Айнур & СемГУ  & Семей & 4 & 3 & 0 & 2 & 0 & 0 & 0 & 5 & \\ 
\hline
\end{tabular}

\newpage

\begin{tabular}{|l|l|l|l|c|*{6}{p{0.3cm}|}c|c|}
\hline
21 & Аханова Алтынай & СемГУ  & Семей & 4 & 2 & 0 & 0 & 1 & 1 & 1 & 5 & \\ 
\hline
22 & Аманғосов Әділет & АРГУ  & Актобе & 4 & 5 & 0 & 0 & 0 & 0 & 0 & 5 & \\ 
\hline
23 & Жумадиллаев Бауыржан & ЮКГУ  & Шымкент & 3 & 3 & 0 & 0 & 0 & 1 & 0 & 4 & \\ 
\hline
24 & Егимбаева Кызгалдак & ПГУ  & Павлодар & 3 & 3 & 0 & 0 & 0 & 0 & 1 & 4 & \\ 
\hline
25 & Жумагазинова Айым & СемГУ  & Семей & 4 & 4 & 0 & 0 & 0 & 0 & 0 & 4 & \\ 
\hline
26 & Орынтаева Умитжан & КарГУ  & Караганда & 3 & 3 & 0 & 0 & 0 & 0 & 0 & 3 & \\ 
\hline
27 & Куттымуратова Фариза & КарГУ  & Караганда & 2 & 3 & 0 & 0 & 0 & 0 & 0 & 3 & \\ 
\hline
28 & Рахметов Әділет & АРГУ  & Актобе & 2 & 2 & 0 & 0 & 0 & 0 & 1 & 3 & \\ 
\hline
29 & Матенов Нартай & ЮКГУ  & Шымкент & 4 & 0 & 0 & 0 & 0 & 1 & 1 & 2 & \\ 
\hline
30 & Калидолдай Айтолқын & ПГУ  & Павлодар & 3 & 0 & 0 & 0 & 0 & 0 & 1 & 1 & \\ 
\hline
\end{tabular}

\newpage

\header{Республиканская олимпиада по МКМ 2016}{01 апреля 2016}
\input{result/2016-rep-mcm-res}
\newpage

\header{Республиканская олимпиада по математике 2017}{13 апреля 2017}
\input{result/2017-rep-math-res}
\newpage

\end{scriptsize}
\end{landscape}

\end{document}