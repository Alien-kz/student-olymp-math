\documentclass[12pt, a5paper]{article}
\usepackage[utf8]{inputenc}
\usepackage[T2A]{fontenc}
\usepackage[russian]{babel}
\usepackage{amsmath}
\usepackage{amsfonts}
\usepackage{amssymb}
\usepackage[left=2cm,right=1cm,top=2cm,bottom=2cm, twoside]{geometry}
\usepackage{lscape}
\usepackage{comment}

\usepackage{pgf,tikz}
\usetikzlibrary{arrows}

\binoppenalty=10000
\relpenalty=10000	

\newcommand{\header}[2]
{
	\subsection{{#1}}
	\begin{center}
%	\textbf{#1}\\
	#2
	\end{center}
}

\sloppy

\graphicspath{{solutions/}}

%space from foothead
\headsep=12pt
\setlength{\headheight}{28pt} 

%header


%center sections name
\usepackage[explicit]{titlesec}
\titleformat{\section}{\normalfont\bfseries\center}{}{1em}{#1}
\titleformat{\subsection}{\normalfont\bfseries\center}{}{1em}{#1}


\renewpagestyle{headings}{
\sethead[\subsectiontitle][][]{}{}{\sectiontitle}
\headrule
\setfoot[\thepage][][]{}{}{\thepage}
}

\begin{document}

\begin{titlepage}
\begin{center}
\vfill

Московский государственный университет\\
имени М.В.~Ломоносова\\
Казахстанский филиал\\

\vfill

{\large\bf 
Студенческие олимпиады по математике\\
Казахстанского филиала МГУ.\\
Задачи и указания. \\
2008--2018 гг.\\}

\vfill

Астана\\
2018
\end{center}
\end{titlepage}

\setcounter{page}{2}


\thispagestyle{empty}

\noindent{\bf УДК} \\
{\bf ББК}\\
{\bf Л}
\vspace{0.7 cm}


{\bf ISBN}  

\vspace{0.7 cm}

В настоящем сборнике представлены более 100 задач студенческих олимпиад по математике Казахстанского филиала МГУ имени М.В.Ломоносова за период с 2008 по 2018 год.

Сборник  адресован  всем интересующимся олимпиадным движением.

\vspace{0.5 cm}

{\bf Составители:}

Абдикалыков А.К., Баев А.Ж., Васильев А.Н.

\vspace{0.5 cm}


{\bf TeX--верстка:}

Баев А.Ж.

\newpage
\begin{small}
\begin{center}
\textbf{Предисловие}
\end{center}
В Казахстанском филиале  МГУ имени М.В.~Ломоносова ежегодно проводятся студенческие олимпиады по математике. На базе Казахстанского филиала проводится также Республиканский этап студенческой предметной олимпиады по математике (2016 и 2017~гг.). Проведение олимпиады в Филиале (уже во второй раз) является результатом побед наших студентов на данной олимпиаде в предыдущие годы.

Ежегодно в декабре студенты Казахстанского филиала проходят отбор внутри университета для участия на Республиканском этапе. В связи со спецификой обучения в Филиале, на олимпиаде участвуют студенты первого и второго курсов механико--математического факультета и факультета вычислительной математики и кибернетики (студенты третьего и четвертого курса продолжают обучение в Москве). На олимпиаде обычно предлагается от 7 до 10 заданий по математическому анализу, алгебре, геометрии, теории чисел, дискретной математике и по крайней мере одна задача из коллекции школьных олимпиад по математике. Такой вариант позволяет первокурсникам свободно конкурировать со студентами второго курса. Победители справляются, как правило, с 4 или 5 заданиями.

В данном сборнике представлены тексты около 100 задач олимпиад последних десяти лет, к которым даны указания, составленные преподавателями Казахстанского филиала. При подготовке олимпиад использованы материалы из различных сборников студенческих и школьных олимпиад, а также авторские задачи. Авторами задач являются преподаватели филиала Абдикалыков А.К., Баев А.Ж., Васильев А.Н., которые активно участвуют в олимпиадном движении Казахстана. 

Составители выражают благодарность профессору  Нурсултанову~Е.Д. и доценту Бекмаганбетову~К.А. за оказание содействия при подготовке сборника.

\begin{flushright}
Директор Казахстанского филиала

А.В.Сидорович
\end{flushright}
\end{small}

\newpage

\mbox{}

\newpage
\pagestyle{headings}

\tableofcontents

\newpage

\section{Условия задач олимпиад Казахстанского филиала}

\header{2008--2009}{7 декабря 2008}
\input{problems/2008}
\newpage

\header{2009--2010}{6 декабря 2009}
\input{problems/2009}
\newpage

\header{2010--2011}{12 декабря 2010}
\input{problems/2010}
\newpage

\header{2011--2012}{10 декабря 2011}
\input{problems/2011}
\newpage

\header{2012--2013}{21 декабря 2012}
\input{problems/2012}
\newpage

\header{2013--2014}{20 декабря 2013}
\input{problems/2013}
\newpage

\header{2013--2014 (дополнительный тур)}{15 марта 2014}
\begin{center}
\begin{tabular}{|l|l|l|c|*{10}{p{0.3cm}|}c|c|}
\hline
№ & Участник & Факультет & Курс & 1 & 2 & 3 & 4 & 5 & 6 & 7 & 8 & 9 & 10 & $\Sigma$\\
\hline
1-2 & Амир Мирас & ВМК & 1 & 9 & 10 &  & 9 &  &  & 10 &  &  &  & 38\\
\hline
1-2 & Шокетаева Надира & ММ & 1 & 9 & 10 & 10 & 9 &  &  &  &  &  &  & 38\\
\hline
3 & Таскынов Ануар & ВМК & 1 & 10 & 10 &  &  &  & 10 &  &  &  &  & 30\\
\hline
4 & Байгабулов Едильхан & Эконом  &2 & 9 & 10 & 10 &  &  &  &  &  &  &  & 29\\
\hline
5 & Тлеубаев Адиль & ВМК & 2 & 9 & 10 &  &  &  &  &  &  &  &  & 19\\
\hline
6 & Нургалиев Мохаммедали & ВМК & 2 &  & 10 &  & 8 &  &  &  &  &  &  & 18\\
\hline
7 & Журавлев Вадим & ВМК & 2 &  & 8 & 9 &  &  &  &  &  &  &  & 17\\
\hline
8 & Тубалыков Кайрат & ММ & 2 & 9 &  &  &  &  &  &  &  &  &  & 9\\
\hline
\end{tabular}
\end{center}
\newpage

\header{2014--2015}{10 декабря 2014}
\begin{center}
\begin{tabular}{|l|l|l|c|*{10}{p{0.3cm}|}c|c|}
\hline
№ & Участник & Спец & Курс & 1 & 2 & 3 & 4 & 5 & 6 & 7 & 8 & 9 & 10 & $\Sigma$ & Диплом\\
\hline
1 & Амир Мирас &  ВМК & 2 & 0 & 10 & 0 & 9 & 10 & 10 & 0 & 0 & 0 & 0 & 39 & 1\\
\hline
2 & Журавская Александра &  ВМК & 1 & 0 & 0 & 0 & 0 & 10 & 10 & 10 & 0 & 8 & 0 & 38 & 1\\
\hline
3 & Булгаков Анатолий &  ВМК & 2 & 2 & 2 & 0 & 0 & 10 & 9 & 0 & 0 & 0 & 0 & 23 & 3\\
\hline
4 & Шокетаева Надира &  ММ & 2 & 3 & 0 & 0 & 0 & 5 & 10 & 0 & 0 & 0 & 0 & 18 & 3\\
\hline
5 & Абайулы Ерулан &  ВМК & 1 & 0 & 0 & 0 & 0 & 10 & 7 & 0 & 0 & 0 & 0 & 17 & 3\\
\hline
6-8 & Таскынов Ануар &  ВМК & 2 & 0 & 0 & 0 & 0 & 0 & 10 & 0 & 0 & 0 & 0 & 10 & \\
\hline
6-8 & Токтаганов Адильхан &  ММ & 1 & 0 & 0 & 0 & 0 & 0 & 10 & 0 & 0 & 0 & 0 & 10 & \\
\hline
6-8 & Таранов Денис &  ВМ & 2 & 0 & 0 & 0 & 0 & 9 & 0 & 0 & 1 & 0 & 0 & 10 & \\
\hline
9 & Батырбеков Аскар &  ММ & 1 & 0 & 0 & 0 & 0 & 0 & 9 & 0 & 0 & 0 & 0 & 9 & \\
\hline
10 & Кенесова Аида &  ВМ & 1 & 0 & 0 & 0 & 0 & 0 & 2 & 0 & 0 & 0 & 0 & 2 & \\
\hline
11 & Даку Ансар &  ВМ & 1 & 0 & 0 & 0 & 0 & 1 & 0 & 0 & 0 & 0 & 0 & 1 & \\
\hline
\end{tabular}

\end{center}
\newpage

\header{2015--2016}{19 декабря 2015}
\input{problems/2015}
\newpage

\header{2016--2017}{10 декабря 2016}
\begin{flushright}
Стоимость задач: \\
10 баллов каждая задача.\\
\end{flushright}

\begin{enumerate}
\item Введём функцию
$$
f(n) = \bigl[\sqrt 1\,\bigr] + \bigl[\sqrt 2\,\bigr] + \bigl[\sqrt 3\,\bigr] + \hdots + \bigl[\sqrt{n^2-1}\,\bigr] + \bigl[\sqrt{n^2}\,\bigr],
$$
где $[x]$ --- наибольшее целое число, не превышающее $x$. Опишите функцию, которая вычисляет $f(n)$ для данного натурального $n$, не используя при этом операцию извлечения корня и вещественную арифметику.

\item На декартовой координатной плоскости нарисованы две полупараболы: график функции $y = x^2$ $(x \geqslant 0)$ и его копия, повёрнутая на прямой угол по часовой стрелке. Эти две кривые отсекают от прямой, параллельной оси ординат, отрезок длины $L$. Обозначим через $S(L)$ --- площадь отсечённой фигуры.

a) Докажите, что $S(L) > 1$ при $L > 2$;

б) Напишите функцию, которая вычисляет $S(L)$ для данного положительного вещественного числа $L$.



\item Найдите все дифференцируемые функции $f\colon \mathbb R\rightarrow\mathbb R$, удовлетворяющие соотношению
$$
f(x-y) + f(x+y) = f'(x^2 + y^2)
$$
для любых $x,y\in\mathbb R$.



\item  Дана функция $f\colon [0, 2n]\rightarrow\mathbb R$. Пусть $f_i = f(i)$ --- значения функции во всех целых $i$ от 0 до $2n$. Дана переменная $S$ вещественного типа с начальным значением 0. За один ход робот может выбрать целое $i$ от 1 до $2n-1$, затем добавить к переменной $S$ или вычесть из нее среднее арифметическое значений функции $f(x)$ в узлах~$i-1$,~$i$,~$i+1$:
$$S := S \pm \frac{f_{i-1}+f_i+f_{i+1}}{3}.$$ 
Может ли робот за конечное число ходов получить в переменной $S$ значение
$$I = \frac{1}{3} \left(f_0 + 4\sum\limits_{k=1}^{n}{f_{2k-1}} + 2\sum\limits_{k=1}^{n-1}{f_{2k}} + f_{2n}\right),$$
которое является приближением интеграла $\displaystyle \int\limits_0^{2n} f(x)\,dx$, если\\
а) $f(0) = f(2n) = 0$;\\
б) $f(0) \ne 0$, $f(2n) \ne 0$?

\item  Дана некоторая условная машина, состоящая из памяти в $n$ бит и указателя, который в каждый отдельный момент находится над какой-то из этих n ячеек. Перед запуском программы в память записывается некоторое натуральное число $m$ в двоичной системе счисления, а указатель устанавливается над крайним правым (младшим) битом числа. Язык программирования для этой машины состоит из следующих команд:

\begin{center}
\begin{tabular}{|c|c|p{14cm}|}
\hline
{\bf L} & left & сместить указатель налево на одну ячейку, если это возможно, иначе завершить программу\\
\hline
{\bf R} & right & сместить указатель направо на одну ячейку, если это возможно, иначе завершить программу\\
\hline
{\bf C} & change & изменить значение бита в текущей ячейке на противоположное\\
\hline
{\bf A} & again & перейти к выполнению первой команды\\
\hline
{\bf S} & skip & пропустить две следующие команды, если в текущей ячейке 0\\
\hline
{\bf F} & finish & завершить выполнение программы\\
\hline
\end{tabular}
\end{center}

Команды записываются в одну строку и выполняются в последовательном порядке, слева направо. При этом запись программы обязана оканчиваться командой \textbf{A} или \textbf{F}. Напишите для этой абстрактной машины следующие программы:

а) заменить данное число на $(m - 1)$;

б) заменить данное число на $(2^n - m - 1)$;

в) изменить на противоположный его старший (крайний слева) бит.

\textit{Примеры:}

а) программа, обнуляющая все ячейки: \textbf{SSCLA};

б) программа, которая изменяет второй справа бит, если крайний справа бит нулевой: \textbf{SFFLCF}.


\item Из квадратной однородной пластины со стороной 1 случайным образом вырезается квадрат со сторонами, равными $2a$ и параллельными сторонам исходного квадрата. При этом центр квадрата --- это случайная величина, равномерно распределённая по всем допустимым положениям (квадрат со стороной $(1 - 2a)$).

a) Найдите вероятность $p(a)$ того, что центр тяжести полученной фигуры лежит в вырезанной области.

б) Опишите функцию $p(a)$, которая вычисляет указанную вероятность приблизительно, считая при этом, что нам не известен метод нахождения центра тяжести произвольной фигуры, однако мы можем найти центр тяжести конечного множества точек одинаковой массы.

\end{enumerate}

\newpage

\header{2017--2018}{9 декабря 2017}
\begin{center}
\begin{tabular}{|l|l|l|c|c|c|c|c|c|c|c|c|}
\hline
№ & Участник & ВУЗ & Курс & 1 & 2 & 3 & 4 & 5 & 6 & $\Sigma$ & Диплом \\
\hline
1 & Жанбырбаев Есеналы & КБТУ & 2 & 10 & 10 & 10 & 3 & 0 & 10 & 43 & 1 степени \\
\hline
2 & Бекмаганбетов Бекарыс & КФ МГУ & 1 & 10 & 10 & 10 & 2 & 10 & 0 & 42 & 2 степени \\
\hline
3 & Сайланбаев Алибек & НУ & 4 & 10 & 10 & 10 & 2 & 0 & 8 & 40 & 2 степени \\
\hline
4 & Аманкелды Акежан & НУ & 4 & 9 & 9 & 10 & 5 & 0 & 0 & 33 & 3 степени \\
\hline
5 & Жанахметов Султан & НУ & 3 & 10 & 10 & 10 & 2 & 0 & 0 & 32 & 3 степени \\
\hline
6 & Шакиев Александр & МУИТ & 2 & 10 & 9 & 10 & 0 & 0 & 0 & 29 & 3 степени \\
\hline
\end{tabular}
\end{center}
\newpage

\header{2017--2018 (дополнительный тур)}{13 марта 2018}
\begin{enumerate}
\item Найдите определитель матрицы $A$ порядка $n$, где 
$$a_{i j} = \frac{(i+j-2)!}{(i-1)! (j-1)!}$$
для всех $1 \leqslant i \leqslant n$ и $1 \leqslant j \leqslant n$.

\item Найдите сумму ряда
$$\sum_{i=2}^{\infty} \frac{3n^2 - 1}{(n^3 - n)^2}.$$

\item Найдите все непрерывные функции $f(x) : \mathbb{R} \to \mathbb{R}$ такие, что 
$$f(x^2) + f(x) = 2.$$

\item При каких $m$, $n$ правильный $m$-угольник можно разрезать на несколько (два и более) равных правильных $n$-угольников?

\item Дана обратимая выпуклая вниз функция $f(x) \in C(R)$. Известно, что существует такая точка $c$, что $f(c) < c$ и $f'(c) = 1$. Докажите, что существуют такие различные точки $a$ и $b$, что
$$\int\limits_{a}^{b} \left(f(x) + f^{-1}(x)\right) dx = b^2 - a^2.$$

\end{enumerate}
\newpage

\header{Республиканская олимпиада по математике 2016}{01 апреля 2016}
\begin{enumerate}

\item Какие натуральные числа представимы в виде $x^2-y^2+2x+2y$ для некоторых целых $x$ и $y$?

\item Пусть  $\alpha(x)$ --- первая цифра после запятой в десятичной записи числа $2^x$.\\
а) Докажите, что функция $\alpha(x)$ интегрируема по Риману на $[0, 1]$.\\
б) Докажите, что $\displaystyle 3.5 < \int\limits_{0}^{1} \alpha(x) dx < 4.5$.

\item В конечном поле произведение всех ненулевых элементов не равно единице. Докажите, что сумма всех элементов поля равна нулю.

\item В эллипсе с фокусами $F_1$ и $F_2$ проведена хорда $MN$, которая проходит через фокус $F_2$. На прямой $F_1F_2$ выбраны две точки $S$ и $T$ такие, что прямые $SM$ и $TN$ являются касательными к эллипсу. Точка $D$ симметрична $F_2$ относительно прямой $SM$, точка $E$ симметрична $F_2$ относительно $NT$. Прямые $DS$, $TE$ и $MN$ при пересечении образуют треугольник $ABC$ (точка $C$ не лежит на $MN$). Докажите, что:
\begin{itemize}
\item[а)] $CF_2$ --- медиана треугольника $ABC$;
\item[б)] $CF_1$ --- биссектриса треугольника $ABC$.
\end{itemize}

\item Максималист и минималист по очереди вписывают по одному числу в таблицу размера $n \times n$ (последовательно, строчка за строчкой, слева направо и сверху вниз). Каким окажется ранг получившейся матрицы, если максималист изо всех сил старается его максимизировать, а минималист --- минимизировать? (Ответ может зависеть от $n$ и от того, кто делает первый ход.) 
 
\item Функция $f : (1, +\infty) \rightarrow \mathbb{R}$ дифференцируема на всей области определения. Известно, что  $$\displaystyle f'(x) = f\left( \frac{x}{x-1} \right) + f(x)$$ для всех $x > 1$ и $\displaystyle \lim\limits_{x \to \infty} \frac{f'(x)}{e^x} = 2$. Докажите, что $f(2) < 20,\hspace{-2pt}16$.

\end{enumerate}

\newpage

\header{Республиканская олимпиада по МКМ 2016}{01 апреля 2016}
\begin{flushright}
Стоимость задач: \\
10 баллов каждая задача.\\
\end{flushright}

\begin{enumerate}
\item Введём функцию
$$
f(n) = \bigl[\sqrt 1\,\bigr] + \bigl[\sqrt 2\,\bigr] + \bigl[\sqrt 3\,\bigr] + \hdots + \bigl[\sqrt{n^2-1}\,\bigr] + \bigl[\sqrt{n^2}\,\bigr],
$$
где $[x]$ --- наибольшее целое число, не превышающее $x$. Опишите функцию, которая вычисляет $f(n)$ для данного натурального $n$, не используя при этом операцию извлечения корня и вещественную арифметику.

\item На декартовой координатной плоскости нарисованы две полупараболы: график функции $y = x^2$ $(x \geqslant 0)$ и его копия, повёрнутая на прямой угол по часовой стрелке. Эти две кривые отсекают от прямой, параллельной оси ординат, отрезок длины $L$. Обозначим через $S(L)$ --- площадь отсечённой фигуры.

a) Докажите, что $S(L) > 1$ при $L > 2$;

б) Напишите функцию, которая вычисляет $S(L)$ для данного положительного вещественного числа $L$.



\item Найдите все дифференцируемые функции $f\colon \mathbb R\rightarrow\mathbb R$, удовлетворяющие соотношению
$$
f(x-y) + f(x+y) = f'(x^2 + y^2)
$$
для любых $x,y\in\mathbb R$.



\item  Дана функция $f\colon [0, 2n]\rightarrow\mathbb R$. Пусть $f_i = f(i)$ --- значения функции во всех целых $i$ от 0 до $2n$. Дана переменная $S$ вещественного типа с начальным значением 0. За один ход робот может выбрать целое $i$ от 1 до $2n-1$, затем добавить к переменной $S$ или вычесть из нее среднее арифметическое значений функции $f(x)$ в узлах~$i-1$,~$i$,~$i+1$:
$$S := S \pm \frac{f_{i-1}+f_i+f_{i+1}}{3}.$$ 
Может ли робот за конечное число ходов получить в переменной $S$ значение
$$I = \frac{1}{3} \left(f_0 + 4\sum\limits_{k=1}^{n}{f_{2k-1}} + 2\sum\limits_{k=1}^{n-1}{f_{2k}} + f_{2n}\right),$$
которое является приближением интеграла $\displaystyle \int\limits_0^{2n} f(x)\,dx$, если\\
а) $f(0) = f(2n) = 0$;\\
б) $f(0) \ne 0$, $f(2n) \ne 0$?

\item  Дана некоторая условная машина, состоящая из памяти в $n$ бит и указателя, который в каждый отдельный момент находится над какой-то из этих n ячеек. Перед запуском программы в память записывается некоторое натуральное число $m$ в двоичной системе счисления, а указатель устанавливается над крайним правым (младшим) битом числа. Язык программирования для этой машины состоит из следующих команд:

\begin{center}
\begin{tabular}{|c|c|p{14cm}|}
\hline
{\bf L} & left & сместить указатель налево на одну ячейку, если это возможно, иначе завершить программу\\
\hline
{\bf R} & right & сместить указатель направо на одну ячейку, если это возможно, иначе завершить программу\\
\hline
{\bf C} & change & изменить значение бита в текущей ячейке на противоположное\\
\hline
{\bf A} & again & перейти к выполнению первой команды\\
\hline
{\bf S} & skip & пропустить две следующие команды, если в текущей ячейке 0\\
\hline
{\bf F} & finish & завершить выполнение программы\\
\hline
\end{tabular}
\end{center}

Команды записываются в одну строку и выполняются в последовательном порядке, слева направо. При этом запись программы обязана оканчиваться командой \textbf{A} или \textbf{F}. Напишите для этой абстрактной машины следующие программы:

а) заменить данное число на $(m - 1)$;

б) заменить данное число на $(2^n - m - 1)$;

в) изменить на противоположный его старший (крайний слева) бит.

\textit{Примеры:}

а) программа, обнуляющая все ячейки: \textbf{SSCLA};

б) программа, которая изменяет второй справа бит, если крайний справа бит нулевой: \textbf{SFFLCF}.


\item Из квадратной однородной пластины со стороной 1 случайным образом вырезается квадрат со сторонами, равными $2a$ и параллельными сторонам исходного квадрата. При этом центр квадрата --- это случайная величина, равномерно распределённая по всем допустимым положениям (квадрат со стороной $(1 - 2a)$).

a) Найдите вероятность $p(a)$ того, что центр тяжести полученной фигуры лежит в вырезанной области.

б) Опишите функцию $p(a)$, которая вычисляет указанную вероятность приблизительно, считая при этом, что нам не известен метод нахождения центра тяжести произвольной фигуры, однако мы можем найти центр тяжести конечного множества точек одинаковой массы.

\end{enumerate}

\newpage

\header{Республиканская олимпиада по математике 2017}{13 апреля 2017}
\begin{enumerate}

\item Пусть $(T_n)_{n=1}^{\infty}$ --- последовательность натуральных чисел, заданная рекуррентно: $T_1 = T_2 = T_3 = 1$ и $T_{n+3} = T_{n+2} + T_{n+1} + T_n$ при $n \geqslant 1$. Вычислите сумму ряда $$\displaystyle \sum\limits_{n=1}^{\infty} \frac{T_n}{2^n},$$
если известно, что данный ряд сходится.

\item Найдите все простые $p$, запись которых в $k$-ичной системе счисления при некотором натуральном $k > 1$ содержит ровно $k$ различных цифр (старшая цифра не может быть нулём).

\item Докажите, что в любой группе квадрат произведения двух элементов порядка два и куб произведения двух элементов порядка три всегда являются коммутаторами.

\item Точка $P$ лежит внутри выпуклой области, ограниченной параболой $y = x^2$, но не лежит на оси $OY$. Обозначим через $S(P)$ множество всех точек, полученных отражением $P$ относительно всех касательных к параболе.

а) Докажите, что значение суммы 
$$\displaystyle\max_{(x, y) \in S(P)} y ~ + \min_{(x, y) \in S(P)} y$$ не зависит от выбора точки $P$.

б) Найдите геометрическое место точек $P$ таких, что $\displaystyle\max_{(x, y) \in S(P)} y = 0.$

\item Для каждой функции $f: [0, 1] \to \mathbb{R}$ обозначим через $s_n(f)$ и $S_n(f)$ нижнюю и верхнюю суммы Дарбу для функции $f$, соответствующие равномерному разбиению $[0, 1]$ на $n$ частей. Существует ли такая интегрируемая функция $f$, что $\displaystyle \sum_{n=1}^{\infty} s_n(f)$ сходится, а
$\displaystyle\sum_{n=1}^{\infty} S_n(f)$ расходится?

\item Некоторые участники математической олимпиады списали решения некоторых задач у своих товарищей. Докажите, что можно с позором выгнать часть участников так, чтобы получилось, что более четверти от общего числа списанных решений было списано выгнанными участниками у не выгнанных.

\end{enumerate}

\newpage

\mbox{}

\newpage

\section{Указания}

\header{2008--2009}{7 декабря 2008}
\input{solutions/2008}
\newpage

\header{2009--2010}{6 декабря 2009}
\input{solutions/2009}
\newpage

\header{2010--2011}{12 декабря 2010}
\input{solutions/2010}
\newpage

\header{2011--2012}{10 декабря 2011}
\input{solutions/2011}
\newpage

\header{2012--2013}{21 декабря 2012}
\input{solutions/2012}
\newpage

\header{2013--2014}{20 декабря 2013}
\input{solutions/2013}
\newpage

\header{2013--2014 (дополнительный тур)}{15 марта 2014}
\begin{center}
\begin{tabular}{|l|l|l|c|*{10}{p{0.3cm}|}c|c|}
\hline
№ & Участник & Факультет & Курс & 1 & 2 & 3 & 4 & 5 & 6 & 7 & 8 & 9 & 10 & $\Sigma$\\
\hline
1-2 & Амир Мирас & ВМК & 1 & 9 & 10 &  & 9 &  &  & 10 &  &  &  & 38\\
\hline
1-2 & Шокетаева Надира & ММ & 1 & 9 & 10 & 10 & 9 &  &  &  &  &  &  & 38\\
\hline
3 & Таскынов Ануар & ВМК & 1 & 10 & 10 &  &  &  & 10 &  &  &  &  & 30\\
\hline
4 & Байгабулов Едильхан & Эконом  &2 & 9 & 10 & 10 &  &  &  &  &  &  &  & 29\\
\hline
5 & Тлеубаев Адиль & ВМК & 2 & 9 & 10 &  &  &  &  &  &  &  &  & 19\\
\hline
6 & Нургалиев Мохаммедали & ВМК & 2 &  & 10 &  & 8 &  &  &  &  &  &  & 18\\
\hline
7 & Журавлев Вадим & ВМК & 2 &  & 8 & 9 &  &  &  &  &  &  &  & 17\\
\hline
8 & Тубалыков Кайрат & ММ & 2 & 9 &  &  &  &  &  &  &  &  &  & 9\\
\hline
\end{tabular}
\end{center}
\newpage

\header{2014--2015}{10 декабря 2014}
\begin{center}
\begin{tabular}{|l|l|l|c|*{10}{p{0.3cm}|}c|c|}
\hline
№ & Участник & Спец & Курс & 1 & 2 & 3 & 4 & 5 & 6 & 7 & 8 & 9 & 10 & $\Sigma$ & Диплом\\
\hline
1 & Амир Мирас &  ВМК & 2 & 0 & 10 & 0 & 9 & 10 & 10 & 0 & 0 & 0 & 0 & 39 & 1\\
\hline
2 & Журавская Александра &  ВМК & 1 & 0 & 0 & 0 & 0 & 10 & 10 & 10 & 0 & 8 & 0 & 38 & 1\\
\hline
3 & Булгаков Анатолий &  ВМК & 2 & 2 & 2 & 0 & 0 & 10 & 9 & 0 & 0 & 0 & 0 & 23 & 3\\
\hline
4 & Шокетаева Надира &  ММ & 2 & 3 & 0 & 0 & 0 & 5 & 10 & 0 & 0 & 0 & 0 & 18 & 3\\
\hline
5 & Абайулы Ерулан &  ВМК & 1 & 0 & 0 & 0 & 0 & 10 & 7 & 0 & 0 & 0 & 0 & 17 & 3\\
\hline
6-8 & Таскынов Ануар &  ВМК & 2 & 0 & 0 & 0 & 0 & 0 & 10 & 0 & 0 & 0 & 0 & 10 & \\
\hline
6-8 & Токтаганов Адильхан &  ММ & 1 & 0 & 0 & 0 & 0 & 0 & 10 & 0 & 0 & 0 & 0 & 10 & \\
\hline
6-8 & Таранов Денис &  ВМ & 2 & 0 & 0 & 0 & 0 & 9 & 0 & 0 & 1 & 0 & 0 & 10 & \\
\hline
9 & Батырбеков Аскар &  ММ & 1 & 0 & 0 & 0 & 0 & 0 & 9 & 0 & 0 & 0 & 0 & 9 & \\
\hline
10 & Кенесова Аида &  ВМ & 1 & 0 & 0 & 0 & 0 & 0 & 2 & 0 & 0 & 0 & 0 & 2 & \\
\hline
11 & Даку Ансар &  ВМ & 1 & 0 & 0 & 0 & 0 & 1 & 0 & 0 & 0 & 0 & 0 & 1 & \\
\hline
\end{tabular}

\end{center}
\newpage

\header{2015--2016}{19 декабря 2015}
\input{solutions/2015}
\newpage

\header{2016--2017}{10 декабря 2016}
\begin{flushright}
Стоимость задач: \\
10 баллов каждая задача.\\
\end{flushright}

\begin{enumerate}
\item Введём функцию
$$
f(n) = \bigl[\sqrt 1\,\bigr] + \bigl[\sqrt 2\,\bigr] + \bigl[\sqrt 3\,\bigr] + \hdots + \bigl[\sqrt{n^2-1}\,\bigr] + \bigl[\sqrt{n^2}\,\bigr],
$$
где $[x]$ --- наибольшее целое число, не превышающее $x$. Опишите функцию, которая вычисляет $f(n)$ для данного натурального $n$, не используя при этом операцию извлечения корня и вещественную арифметику.

\item На декартовой координатной плоскости нарисованы две полупараболы: график функции $y = x^2$ $(x \geqslant 0)$ и его копия, повёрнутая на прямой угол по часовой стрелке. Эти две кривые отсекают от прямой, параллельной оси ординат, отрезок длины $L$. Обозначим через $S(L)$ --- площадь отсечённой фигуры.

a) Докажите, что $S(L) > 1$ при $L > 2$;

б) Напишите функцию, которая вычисляет $S(L)$ для данного положительного вещественного числа $L$.



\item Найдите все дифференцируемые функции $f\colon \mathbb R\rightarrow\mathbb R$, удовлетворяющие соотношению
$$
f(x-y) + f(x+y) = f'(x^2 + y^2)
$$
для любых $x,y\in\mathbb R$.



\item  Дана функция $f\colon [0, 2n]\rightarrow\mathbb R$. Пусть $f_i = f(i)$ --- значения функции во всех целых $i$ от 0 до $2n$. Дана переменная $S$ вещественного типа с начальным значением 0. За один ход робот может выбрать целое $i$ от 1 до $2n-1$, затем добавить к переменной $S$ или вычесть из нее среднее арифметическое значений функции $f(x)$ в узлах~$i-1$,~$i$,~$i+1$:
$$S := S \pm \frac{f_{i-1}+f_i+f_{i+1}}{3}.$$ 
Может ли робот за конечное число ходов получить в переменной $S$ значение
$$I = \frac{1}{3} \left(f_0 + 4\sum\limits_{k=1}^{n}{f_{2k-1}} + 2\sum\limits_{k=1}^{n-1}{f_{2k}} + f_{2n}\right),$$
которое является приближением интеграла $\displaystyle \int\limits_0^{2n} f(x)\,dx$, если\\
а) $f(0) = f(2n) = 0$;\\
б) $f(0) \ne 0$, $f(2n) \ne 0$?

\item  Дана некоторая условная машина, состоящая из памяти в $n$ бит и указателя, который в каждый отдельный момент находится над какой-то из этих n ячеек. Перед запуском программы в память записывается некоторое натуральное число $m$ в двоичной системе счисления, а указатель устанавливается над крайним правым (младшим) битом числа. Язык программирования для этой машины состоит из следующих команд:

\begin{center}
\begin{tabular}{|c|c|p{14cm}|}
\hline
{\bf L} & left & сместить указатель налево на одну ячейку, если это возможно, иначе завершить программу\\
\hline
{\bf R} & right & сместить указатель направо на одну ячейку, если это возможно, иначе завершить программу\\
\hline
{\bf C} & change & изменить значение бита в текущей ячейке на противоположное\\
\hline
{\bf A} & again & перейти к выполнению первой команды\\
\hline
{\bf S} & skip & пропустить две следующие команды, если в текущей ячейке 0\\
\hline
{\bf F} & finish & завершить выполнение программы\\
\hline
\end{tabular}
\end{center}

Команды записываются в одну строку и выполняются в последовательном порядке, слева направо. При этом запись программы обязана оканчиваться командой \textbf{A} или \textbf{F}. Напишите для этой абстрактной машины следующие программы:

а) заменить данное число на $(m - 1)$;

б) заменить данное число на $(2^n - m - 1)$;

в) изменить на противоположный его старший (крайний слева) бит.

\textit{Примеры:}

а) программа, обнуляющая все ячейки: \textbf{SSCLA};

б) программа, которая изменяет второй справа бит, если крайний справа бит нулевой: \textbf{SFFLCF}.


\item Из квадратной однородной пластины со стороной 1 случайным образом вырезается квадрат со сторонами, равными $2a$ и параллельными сторонам исходного квадрата. При этом центр квадрата --- это случайная величина, равномерно распределённая по всем допустимым положениям (квадрат со стороной $(1 - 2a)$).

a) Найдите вероятность $p(a)$ того, что центр тяжести полученной фигуры лежит в вырезанной области.

б) Опишите функцию $p(a)$, которая вычисляет указанную вероятность приблизительно, считая при этом, что нам не известен метод нахождения центра тяжести произвольной фигуры, однако мы можем найти центр тяжести конечного множества точек одинаковой массы.

\end{enumerate}

\newpage

\header{2017--2018}{9 декабря 2017}
\begin{center}
\begin{tabular}{|l|l|l|c|c|c|c|c|c|c|c|c|}
\hline
№ & Участник & ВУЗ & Курс & 1 & 2 & 3 & 4 & 5 & 6 & $\Sigma$ & Диплом \\
\hline
1 & Жанбырбаев Есеналы & КБТУ & 2 & 10 & 10 & 10 & 3 & 0 & 10 & 43 & 1 степени \\
\hline
2 & Бекмаганбетов Бекарыс & КФ МГУ & 1 & 10 & 10 & 10 & 2 & 10 & 0 & 42 & 2 степени \\
\hline
3 & Сайланбаев Алибек & НУ & 4 & 10 & 10 & 10 & 2 & 0 & 8 & 40 & 2 степени \\
\hline
4 & Аманкелды Акежан & НУ & 4 & 9 & 9 & 10 & 5 & 0 & 0 & 33 & 3 степени \\
\hline
5 & Жанахметов Султан & НУ & 3 & 10 & 10 & 10 & 2 & 0 & 0 & 32 & 3 степени \\
\hline
6 & Шакиев Александр & МУИТ & 2 & 10 & 9 & 10 & 0 & 0 & 0 & 29 & 3 степени \\
\hline
\end{tabular}
\end{center}
\newpage

\header{2017--2018 (дополнительный тур)}{13 марта 2018}
\begin{enumerate}
\item Найдите определитель матрицы $A$ порядка $n$, где 
$$a_{i j} = \frac{(i+j-2)!}{(i-1)! (j-1)!}$$
для всех $1 \leqslant i \leqslant n$ и $1 \leqslant j \leqslant n$.

\item Найдите сумму ряда
$$\sum_{i=2}^{\infty} \frac{3n^2 - 1}{(n^3 - n)^2}.$$

\item Найдите все непрерывные функции $f(x) : \mathbb{R} \to \mathbb{R}$ такие, что 
$$f(x^2) + f(x) = 2.$$

\item При каких $m$, $n$ правильный $m$-угольник можно разрезать на несколько (два и более) равных правильных $n$-угольников?

\item Дана обратимая выпуклая вниз функция $f(x) \in C(R)$. Известно, что существует такая точка $c$, что $f(c) < c$ и $f'(c) = 1$. Докажите, что существуют такие различные точки $a$ и $b$, что
$$\int\limits_{a}^{b} \left(f(x) + f^{-1}(x)\right) dx = b^2 - a^2.$$

\end{enumerate}
\newpage
%%%%%%%%%%%%%%%%%%%%

\header{Республиканская олимпиада по математике 2016}{01 апреля 2016}
\begin{flushright}
Стоимость задач: \\
10 баллов каждая задача.\\
\end{flushright}

\begin{enumerate}
\item Введём функцию
$$
f(n) = \bigl[\sqrt 1\,\bigr] + \bigl[\sqrt 2\,\bigr] + \bigl[\sqrt 3\,\bigr] + \hdots + \bigl[\sqrt{n^2-1}\,\bigr] + \bigl[\sqrt{n^2}\,\bigr],
$$
где $[x]$ --- наибольшее целое число, не превышающее $x$. Опишите функцию, которая вычисляет $f(n)$ для данного натурального $n$, не используя при этом операцию извлечения корня и вещественную арифметику.

\item На декартовой координатной плоскости нарисованы две полупараболы: график функции $y = x^2$ $(x \geqslant 0)$ и его копия, повёрнутая на прямой угол по часовой стрелке. Эти две кривые отсекают от прямой, параллельной оси ординат, отрезок длины $L$. Обозначим через $S(L)$ --- площадь отсечённой фигуры.

a) Докажите, что $S(L) > 1$ при $L > 2$;

б) Напишите функцию, которая вычисляет $S(L)$ для данного положительного вещественного числа $L$.



\item Найдите все дифференцируемые функции $f\colon \mathbb R\rightarrow\mathbb R$, удовлетворяющие соотношению
$$
f(x-y) + f(x+y) = f'(x^2 + y^2)
$$
для любых $x,y\in\mathbb R$.



\item  Дана функция $f\colon [0, 2n]\rightarrow\mathbb R$. Пусть $f_i = f(i)$ --- значения функции во всех целых $i$ от 0 до $2n$. Дана переменная $S$ вещественного типа с начальным значением 0. За один ход робот может выбрать целое $i$ от 1 до $2n-1$, затем добавить к переменной $S$ или вычесть из нее среднее арифметическое значений функции $f(x)$ в узлах~$i-1$,~$i$,~$i+1$:
$$S := S \pm \frac{f_{i-1}+f_i+f_{i+1}}{3}.$$ 
Может ли робот за конечное число ходов получить в переменной $S$ значение
$$I = \frac{1}{3} \left(f_0 + 4\sum\limits_{k=1}^{n}{f_{2k-1}} + 2\sum\limits_{k=1}^{n-1}{f_{2k}} + f_{2n}\right),$$
которое является приближением интеграла $\displaystyle \int\limits_0^{2n} f(x)\,dx$, если\\
а) $f(0) = f(2n) = 0$;\\
б) $f(0) \ne 0$, $f(2n) \ne 0$?

\item  Дана некоторая условная машина, состоящая из памяти в $n$ бит и указателя, который в каждый отдельный момент находится над какой-то из этих n ячеек. Перед запуском программы в память записывается некоторое натуральное число $m$ в двоичной системе счисления, а указатель устанавливается над крайним правым (младшим) битом числа. Язык программирования для этой машины состоит из следующих команд:

\begin{center}
\begin{tabular}{|c|c|p{14cm}|}
\hline
{\bf L} & left & сместить указатель налево на одну ячейку, если это возможно, иначе завершить программу\\
\hline
{\bf R} & right & сместить указатель направо на одну ячейку, если это возможно, иначе завершить программу\\
\hline
{\bf C} & change & изменить значение бита в текущей ячейке на противоположное\\
\hline
{\bf A} & again & перейти к выполнению первой команды\\
\hline
{\bf S} & skip & пропустить две следующие команды, если в текущей ячейке 0\\
\hline
{\bf F} & finish & завершить выполнение программы\\
\hline
\end{tabular}
\end{center}

Команды записываются в одну строку и выполняются в последовательном порядке, слева направо. При этом запись программы обязана оканчиваться командой \textbf{A} или \textbf{F}. Напишите для этой абстрактной машины следующие программы:

а) заменить данное число на $(m - 1)$;

б) заменить данное число на $(2^n - m - 1)$;

в) изменить на противоположный его старший (крайний слева) бит.

\textit{Примеры:}

а) программа, обнуляющая все ячейки: \textbf{SSCLA};

б) программа, которая изменяет второй справа бит, если крайний справа бит нулевой: \textbf{SFFLCF}.


\item Из квадратной однородной пластины со стороной 1 случайным образом вырезается квадрат со сторонами, равными $2a$ и параллельными сторонам исходного квадрата. При этом центр квадрата --- это случайная величина, равномерно распределённая по всем допустимым положениям (квадрат со стороной $(1 - 2a)$).

a) Найдите вероятность $p(a)$ того, что центр тяжести полученной фигуры лежит в вырезанной области.

б) Опишите функцию $p(a)$, которая вычисляет указанную вероятность приблизительно, считая при этом, что нам не известен метод нахождения центра тяжести произвольной фигуры, однако мы можем найти центр тяжести конечного множества точек одинаковой массы.

\end{enumerate}

\newpage

\header{Республиканская олимпиада по МКМ 2016}{01 апреля 2016}
\begin{enumerate}

\item (Абдикалыков А.К.)

Максимальный балл давался за алгоритм с асимптотической сложностью $O(1)$ (явную формулу), промежуточные баллы --- за сложность $O(n)$ и $O(n^2)$.
\begin{multline*}
\sum\limits_{j=1}^{n^2}{[\sqrt j\,]}=
\sum\limits_{k=1}^{n}{\left(k\cdot\sum\limits_{\substack{[\sqrt j\,]=k \\ 1 \leqslant j \leqslant n^2}}{1}\right)}=
\sum\limits_{k=1}^{n-1}{\left(k\cdot\sum\limits_{\substack{[\sqrt j\,]=k \\ 1 \leqslant j \leqslant n^2}}{1}\right)}+n=\\
=\sum\limits_{k=1}^{n-1}{\left(k\cdot\sum\limits_{j=k^2}^{(k+1)^2-1}{1}\right)}+n=
\sum\limits_{k=1}^{n-1}{k(2k+1)}+n=\\
=\sum\limits_{k=1}^{n-1}{(2k^2+k)}+n=
2\cdot\frac{(n-1)n(2n-1)}{6}+\frac{(n-1)n}{2}+n=\\
=\frac{(n-1)n(4n+1)}{6}+n=
\frac{n(4n^2-3n+5)}{6}
\end{multline*}
    
\item (Абдикалыков А.К.)

а) Две полупараболы из условия задачи --- это графики функций $y=x^2$ и $y=-\sqrt{x}$ при $x\geqslant 0$. Поэтому искомая функция
$$
S(L)=\int\limits_{0}^{f^{-1}(L)}{f(x)~dx},
$$
где $f(x)=x^2+\sqrt{x}$. (В силу монотонности функции $f(x)$ корректно вводить $f^{-1}(L)$.) Так как, кроме прочего, подынтегральная функция в определении $S(L)$ положительная, то и сама функция $S(L)$ --- возрастающая. Поскольку $S(2)=1$ (это можно показать разными способами: как графически, составив квадрат, например, так и аналитически, посчитав явно интеграл), то $S(L)>1$ при $L>2$.\\
б) Найдём сначала $x_0=f^{-1}(L)$ с помощью бинарного поиска, затем вычислим
$$
S(L)=\left.\left(\frac{1}{3}x^3+\frac{2}{3}x^{\frac{3}{2}}\right)\right|_{x=0}^{x_0}=
\frac{x_0^3+2x_0^{\frac{3}{2}}}{3}=$$
$$=\frac{x_0(2x_0^2+2\sqrt{x_0}-x_0^2)}{3}=
\frac{x_0(2L-x_0^2)}{3}.
$$

\item (Баев А.Ж.)

Продифференцируем по $x$ и $y$:
$$f'(x - y) + f'(x + y) = 2 x f''(x^2 + y^2),$$
$$ - f'(x - y) + f'(x + y) = 2 y f''(x^2 + y^2).$$

Пусть $x \ne 0$, $y \ne 0$. Приравняем $f''(x^2 + y^2)$:
$$(y + x) f'(x - y) = (x - y) f'(x + y).$$

Пусть $|x| \ne |y|$.
$$\frac{f'(x + y)}{x + y} = \frac{f'(x - y)}{x - y}.$$

Зафиксируем величину $x - y = A$, отличную от нуля. Тогда выражение справа не зависит от $y$ и равно некоторой константе $2C$.
$$\frac{f'(2y + A)}{2y + A} = \frac{f'(A)}{A} = 2 C.$$

Заметим, что $t = 2y + A$ может принимать любые ненулевые значения. Значит, при $t \ne 0$:
$$f'(t) = 2Ct.$$
$$f(t) = Ct^2 + C_1.$$

При подстановке в исходное уравнение, получим: $C_1 = 0$. При $t = 0$ доопределяется из непрерывности $f'(t)$ (по соотношению в условии). Ответ: $f(t) = C t^2$.

\item (Баев А.Ж., Абдикалыков А.К.)

Пусть конечное значение $S=\sum\limits_{j=1}^{2n-1}c_j\cdot\frac{f_{j-1}+f_j+f_{j+1}}{3}$. Тогда пункт б) эквивалентен решению нижеуказанной системы линейных уравнений, причём в целых числах. Видно, что система состоит из $(2n+1)$ уравнения относительно $(2n-1)$ неизвестной.
\begin{equation*}
\begin{pmatrix}
1 & 0 & 0 & \cdots & 0 & 0 & 0 \\
1 & 1 & 0 & \cdots & 0 & 0 & 0 \\
1 & 1 & 1 & \cdots & 0 & 0 & 0 \\
0 & 1 & 1 & \cdots & 0 & 0 & 0 \\
\cdots & \cdots & \cdots & \cdots & \cdots & \cdots & \cdots\\
0 & 0 & 0 & \cdots & 1 & 1 & 0 \\
0 & 0 & 0 & \cdots & 1 & 1 & 1\\
0 & 0 & 0 & \cdots & 0 & 1 & 1\\
0 & 0 & 0 & \cdots & 0 & 0 & 1\\
\end{pmatrix}
\begin{pmatrix}
c_1 \\
c_2 \\
c_3 \\
\cdots \\
c_{2n-3} \\
c_{2n-2} \\
c_{2n-1}
\end{pmatrix}
=
\begin{pmatrix}
1 \\
4 \\
2 \\
4 \\
\cdots \\
4 \\
2 \\
4 \\
1
\end{pmatrix}
\end{equation*}

Рассматривая только первые $(2n-1)$ уравнения, мы получим систему с нижнетреугольной матрицей и определителем, равным единице, а значит, всеми уравнениями, кроме последних двух, все неизвестные определяются однозначно, принимая при этом целые значения. Таким образом, задача сводится к нахождению таких $n$, чтобы система из этих $(2n-1)$ уравнения имела решение, совместимое с дополнительными условиями $c_{2n-2}+c_{2n-1}=4$, $c_{2n-1}=1$. Решая эту систему методом Гаусса, получаем
$$
\begin{matrix}
c_1=1, & c_2=3, & c_3=-2,\\
c_4=3, & c_5=1, & c_6=0,\\
c_7=1, & c_8=3, & c_9=-2,\\
c_{10}=3, & c_{11}=1, & c_{12}=0,\\
\end{matrix}
$$
$$
\hdots
$$
Таким образом, равенства $c_{2n-2}=3$, $c_{2n-1}=1$ выполняются только в том случае, если $$2n-1=5\pmod 6,$$ или, что то же самое, $n$ кратно 3.

Пункт а) эквивалентен решению той же системы в целых числах, но уже без первого и последнего уравнений.
\begin{equation*}
\begin{pmatrix}
1 & 1 & 0 & \cdots & 0 & 0 & 0 \\
1 & 1 & 1 & \cdots & 0 & 0 & 0 \\
0 & 1 & 1 & \cdots & 0 & 0 & 0 \\
\cdots & \cdots & \cdots & \cdots & \cdots & \cdots & \cdots\\
0 & 0 & 0 & \cdots & 1 & 1 & 0 \\
0 & 0 & 0 & \cdots & 1 & 1 & 1\\
0 & 0 & 0 & \cdots & 0 & 1 & 1
\end{pmatrix}
\begin{pmatrix}
c_1 \\
c_2 \\
c_3 \\
\cdots \\
c_{2n-3} \\
c_{2n-2} \\
c_{2n-1}
\end{pmatrix}
=
\begin{pmatrix}
4 \\
2 \\
4 \\
\cdots \\
4 \\
2 \\
4
\end{pmatrix}
\end{equation*}

Фиксируя $c_1=c$ и используя все уравнения, кроме последнего ($c_{2n-2}+c_{2n-1}=4$), находим
$$
\begin{matrix}
c_1=c, & c_2=4-c, & c_3=-2, \\
c_4=2+c, & c_5=2-c, & c_6=0,\\
c_7=c, & c_8=4-c, & c_9=-2, \\
c_{10}=2+c, & c_{11}=2-c, & c_{12}=0,
\end{matrix}
$$
$$
\hdots
$$
Значит,
$$
c_{2n-2}+c_{2n-1}=
\begin{cases}
c, & 2n-2=0\pmod 6,\\
-2-c, & 2n-2=2\pmod 6,\\
4, & 2n-2=4\pmod 6.
\end{cases}
$$
Видно, что в любом случае можно подобрать такое $c$, чтобы выполнялось равенство $$c_{2n-2}+c_{2n-1}=4,$$ из чего следует, что система совместна при любом $n$.

\item  (Абдикалыков А.К.)

а) Уменьшить двоичное число на единицу: {\bf CSLAF}.

б) Поменять все биты: {\bf CLA}.

в) Поменять только старший бит: {\bf CLRCLA}.

\item (Баев А.Ж.)

а) Пусть исходный квадрат --- это квадрат $[0, 1] \times [0, 1]$ на плоскости, а центр вырезанного квадрата расположен в точке $(x_0, y_0)$. Квадрат целиком поместится, если $(x_0; y_0) \in [a, 1 - a] \times [a, 1 - a]$. 

1 шаг. Найдем центр тяжести. Запишем функцию плотности массы пластины по оси $OX$:
$$
f(x) =
\begin{cases}
1, x < x_0 - a\\
1 - 2a, x \in [x_0 - a, x_0 + a]\\
1, x < x_0 + a\\
\end{cases}
$$

Найдем проекцию центра тяжести $m$ на ось $OX$:
$$m = \frac{\int_0^1 x f(x) dx}{\int_0^1 f(x) dx} = \frac{1 - 8 a^2 x_0}{2 (1 - 4 a^2)} .$$

2 шаг. Найдем вероятность попадания центра тяжести в вырезанную часть.
$$P = P(m \in [x_0 - a; x_0 + a]) =$$
$$= P\left( \frac{1 - 8 a^2 x_0}{2 (1 - 4 a^2)} < x_0 + a \right) - P\left( \frac{1 - 8 a^2 x_0}{2 (1 - 4 a^2)} < x_0 - a \right) =$$
$$= P\left( x_0 + a > \frac{1}{2} + 4 a^3  \right) - P\left( x_0 - a > \frac{1}{2} - 4 a^3 \right). $$
Обознаим полученную разность $P_1 - P_2.$

Вычислим $P_1$. Заметим, что $x_0 + a$ равномерно распределено на $[2a, 1]$. Поэтому важно понять, попадает ли $\frac{1}{2} + 4 a^3$ в интервал $[2a, 1]$. $\frac{1}{2} + 4 a^3 < 1$ ввиду того, что $a \in [0, \frac12]$. Проверим левую границу:

$$\frac{1}{2} + 4 a^3 > 2a$$
$$ \left(a - \frac{1}{2} \right)\left(a - \varphi \right)\left(a - \overline{\varphi}\right) > 0$$
где $\varphi= \frac{-1 + \sqrt{5}}{4}$. Значит, $\frac{1}{2} + 4 a^3$ попадает в интервал $[2a, 1]$ при $a < \varphi$.

$$P_1 = 
\begin{cases}
\frac{1 - 8 a^3}{2(1 - 2a)}, &a < \varphi\\
1, &a > \varphi.
\end{cases}
$$

Аналогично вычислим $P_2$. $x_0 - a$ равномерно распределено на $[0, 1-2a]$. Поэтому важно понять, попадает ли $\frac{1}{2} - 4 a^3$ в интервал $[0, 1-2a]$. $\frac{1}{2} - 4 a^3 > 0$ ввиду того, что $a \in [0, \frac12]$. Значит:
$$P_2 = 
\begin{cases}
 \frac{1 - 4a + 8 a^3}{2(1 - 2a)}, &a < \varphi\\
0, &a > \varphi.
\end{cases}
$$

Найдем вероятность попадания центра тяжести в вырезанную часть:
$$(P_1 - P_2)^2 =
\begin{cases}
4a^2(1+2a)^2,& a \in [0, \frac{-1 + \sqrt{5}}{4}]\\
1,& a \in \left[\frac{-1 + \sqrt{5}}{4}, \frac12 \right]\\
\end{cases}
$$

б) Промоделируем методом Монте--Карло и подсчет центра тяжести, и подсчет ответа. Генерируем $N$ подходящих квадратов. У каждого из них генерируем $M$ случайных точек. Если центр тяжести данных точек находится внутри квадрата, то засчитываем этот квадрат. Иначе --- нет. Отметим, что порядок аппроксимации данного метода $O\left( \frac{1}{ \sqrt{NM} } \right)$.

\end{enumerate} 

\newpage

\header{Республиканская олимпиада по математике 2017}{13 апреля 2017}
\begin{enumerate}

\item (Абдикалыков А.)

Пусть $S=\displaystyle\sum\limits_{n=1}^{\infty}{\frac{T_n}{2^n}}$. Тогда
$$
\sum\limits_{n=1}^{\infty}{\frac{T_{n+1}}{2^n}}=2\cdot \sum\limits_{n=1}^{\infty}{\frac{T_{n+1}}{2^{n+1}}}=2\cdot \left(S-\frac{T_1}{2^1}\right)=2S-1,
$$
$$
\sum\limits_{n=1}^{\infty}{\frac{T_{n+2}}{2^n}}=4\cdot \sum\limits_{n=1}^{\infty}{\frac{T_{n+2}}{2^{n+2}}}=4\cdot\left(S-\frac{T_1}{2^1}-\frac{T_2}{2^2}\right)=4S-3,
$$
\begin{multline*}
\sum\limits_{n=1}^{\infty}{\frac{T_{n+3}}{2^n}}=8\cdot \sum\limits_{n=1}^{\infty}{\frac{T_{n+3}}{2^{n+3}}}=\\
=8\cdot\left(S-\frac{T_1}{2^1}-\frac{T_2}{2^2}-\frac{T_3}{2^3}\right)=8S-7.
\end{multline*}

Так как по условию $T_{n+3}=T_{n+2}+T_{n+1}+T_n$, то $$8S-7=4S-3+2S-1+S,$$ откуда следует $S=3$.

\item (Абдикалыков А.)

Пусть $k$-ичная запись простого числа $p$ для некоторого $k>1$ выглядит как $\overline{a_0a_1\hdots a_{k-1}}$, где $(a_0, a_1, \hdots, a_{k-1})$ --- некоторая перестановка цифр $(0, 1, \hdots, k-1)$. Тогда
$$
p = a_0\cdot k^{k-1} + a_1\cdot k^{k-2}+\hdots+a_{k-1}\cdot k^0 \equiv
$$
$$
= a_0+a_1+\hdots+a_{k-1} \pmod {(k-1)}.
$$
Поскольку сумма всех цифр равна $k(k-1)/2$, то можно сделать вывод, что число $p$ делится на $(k-1)/2$, если $k$ нечётно и на $k-1$, если $k$ чётно. Учитывая, что $p\geqslant k^{k-1}>k-1$ --- простое число, заключаем, что $k$ должно удовлетворять совокупности соотношений
$$
\left[
\begin{matrix}
\frac{k-1}{2}=1,& k = 2l+1,\\
k-1=1, & k = 2l.\\
\end{matrix}
\right.
$$
Таким образом, $k=2$ или $k=3$, а значит, достаточно перебрать числа $10_2$, $102_3$, $120_3$, $201_3$, $210_3$. Простыми среди них являются только $2=10_2$, $11=102_3$ и $19=201_3$.

\item (Клячко А.)

а) Для любого элемента $x$ порядка 2 верно $x=x^{-1}$, поэтому
$$
(ab)^2=abab=aba^{-1}b^{-1},
$$
если $a^2=b^2=e$.

б) Аналогично, для любого элемента $x$ порядка 3 верно $x^2=x^{-1}$, поэтому
\begin{multline*}
(ab)^3=ababab=ab^4aba^4b=(ab^2)(b^2a)(ba^2)(a^2b)=\\
(ab^2)(b^2a)(ab^2)^{-1}(b^2a)^{-1},
\end{multline*}
если $a^3=b^3=e$.

\item (Баев~А.)

Обозначим через $F$ фокус параболы,  через $d$ директрису параболы. Рассмотрим произвольную касательную к параболе $l$ в произвольной точке $C$.

\begin{center}
\includegraphics[scale=0.9]{pictures/2017-republic}
\end{center}

Свойство 1: точка $F_C$, симметричная $F$ относительно $l$, лежит на директрисе $d$. 

Из определения параболы: $FC = F_CC$. Из оптического свойства параболы $\angle(FC; l) = \angle(l; F_CC)$. Получаем, что $l$ --- ось симметрии для отрезков $FC$ и $F_CC$. 

Свойство 2: $$ -\frac{1}{4} - FP \leqslant y(P_C)  \leqslant -\frac{1}{4} + FP,$$
где $y(P_C)$ --- ордината точки $P_C$.

Известна директриса данной параболы $y = -\frac{1}{4}$. Ордината точки $F_C$ равна $-\frac{1}{4}$. А точка $P_C$ находится на расстояния не более, чем $F_CP_C$ от директрисы. Осталось заметить, что с учетом свойства 1 треугольники $FCP$ и $F_CCP_C$ равны, то есть $F_CP_C = FP$.

Свойство 3:  $\displaystyle \max_{(x, y) \in S(P)} y = -\frac{1}{4} + FP$ и $\displaystyle \min_{(x, y) \in S(P)} y = -\frac{1}{4} - FP$. 

Максимум или минимум $y(P_C)$ в свойстве 2 достигается в том случае, если $P_CF_C$ перпендикулярно директрисе. Причем для максимума необходимо, чтобы $P_C$ и $C$ лежали по одну сторону от директрисы, а для минимума --- по разные стороны. То есть угол $CF_CP_C$ равен либо 0, либо $\pi$ (соответственно, угол $CFP$ равен либо 0, либо $\pi$). В качестве таких точек $C$ достаточно выбрать точки  пересечения $FP$ с параболой $A$ и $B$. Значит, оба равенства в свойстве 2 достигаются.

Свойство 4: геометрическим местом точек в пункте б) является окружность с центром в $F$ и радиусом $\frac{1}{4}$. Из свойства 3 следует, что $FP = F_BP_B = \frac{1}{4}$.

\item (Васильев А.)

Ответ: да, существует.

Можно привести множество примеров, но мы укажем самый простой:
$$
f(x) =
\begin{cases}
0, x \in [0, 1)\\
1, x = 1
\end{cases}
$$
Обозначив $\frac{i}{n}$ через $x_i$, имеем: $m_i = \inf_{x \in [x_{i-1}, x_i]} f(x) = 0$ для всех $i = \overline{1,n}$ и 
$$M_i = \sup_{x \in [x_{i-1}, x_i]} f(x) =\begin{cases} 
0, i = \overline{1, n-1} \\ 
1, i = n
\end{cases}
$$
Следовательно, $s_n(f) = 0$ и $S_n(f) = \frac{1}{n}$ для всех $n \in \mathbb{N}$. При этом $f$ интегрируема на $[0, 1]$, ряд $\displaystyle \sum_{n=1}^{\infty} 0$ сходится, а ряд $\displaystyle \sum_{n=1}^{\infty} \frac{1}{n}$ расходится.

\item (Высоканов Б., Клячко А.)

Обозначим через $n$ количество участников олимпиады и присвоим им номера от 1 до $n$. Пусть $a_{ij}$ --- количество решений, списанных $i$-ым участником у $j$-го, при этом полагаем $a_{ii} = 0$. Рассмотрим два случая:

1) $n = 2k$, $k \in \mathbb{N}$. Если $k = 1$, то доказательство тривиально. Пусть $k \leqslant 2$. Доказательство проведем от противного. Допустим, что, выгоняя любые $k$ человек из $2k$, мы никогда не достигнем требуемого. Тогда для любого $S' \subset S$, где $|S'| = k$ и $S = \{1, 2, ..., 2k \}$, имеем:
$$\sum_{\substack{i \in S' \\ j \in S \backslash S'}} \leqslant \frac{1}{4} \sum_{i, j \in S} a_{ij}.$$

Просуммируем эти неравенства по всем $S'$:
$$\sum_{S'} \sum_{\substack{i \in S' \\ j \in S \backslash S'}} a_{ij} \leqslant \frac{1}{4} \sum_{S'} \sum_{i, j \in S} a_{ij}.$$

Заметим, что каждое $a_{ij}$ при $i \neq j$ в сумме слева встретится ровно $C_{2k-2}^{k-1}$ раз, а в сумме справа --- ровно $C_{2k}^{k}$ раз. Разделив обе части неравенства на $\sum_{i, j \in S} a_{ij} > 0$, находим:
$$ C_{2k-2}^{k-1} \leqslant \frac{1}{4} C_{2k}^k,$$
что неверно, так как 
$$\frac{C_{2k}^{k}}{C_{2k-2}^{k-1}} = 2 \left( 2 - \frac{1}{k} \right) < 4.$$

2) $n = 2k + 1$, $k \in \mathbb{N}$. При $k = 1$ доказательство тривиально. При $k \geqslant 2$ рассуждаем аналогично 1), рассматривая все $S' \in S$ с условием $|S'| = k$ (при этом $S = \{1, 2, ..., 2k+1\}$).

\end{enumerate}
\newpage

%%%%%%%%%%%%%%%%%%%%%%%%%%%%%%%%%%%%%%%%%%%%

\begin{landscape}

\section{Результаты}

\begin{scriptsize}


\header{2012--2013}{21 декабря 2012}
\input{results/2012}
\newpage

\header{2013--2014}{20 декабря 2013}
\input{results/2013}
\newpage

\header{2013--2014 (дополнительный тур)}{15 марта 2014}
\begin{center}
\begin{tabular}{|l|l|l|c|*{10}{p{0.3cm}|}c|c|}
\hline
№ & Участник & Факультет & Курс & 1 & 2 & 3 & 4 & 5 & 6 & 7 & 8 & 9 & 10 & $\Sigma$\\
\hline
1-2 & Амир Мирас & ВМК & 1 & 9 & 10 &  & 9 &  &  & 10 &  &  &  & 38\\
\hline
1-2 & Шокетаева Надира & ММ & 1 & 9 & 10 & 10 & 9 &  &  &  &  &  &  & 38\\
\hline
3 & Таскынов Ануар & ВМК & 1 & 10 & 10 &  &  &  & 10 &  &  &  &  & 30\\
\hline
4 & Байгабулов Едильхан & Эконом  &2 & 9 & 10 & 10 &  &  &  &  &  &  &  & 29\\
\hline
5 & Тлеубаев Адиль & ВМК & 2 & 9 & 10 &  &  &  &  &  &  &  &  & 19\\
\hline
6 & Нургалиев Мохаммедали & ВМК & 2 &  & 10 &  & 8 &  &  &  &  &  &  & 18\\
\hline
7 & Журавлев Вадим & ВМК & 2 &  & 8 & 9 &  &  &  &  &  &  &  & 17\\
\hline
8 & Тубалыков Кайрат & ММ & 2 & 9 &  &  &  &  &  &  &  &  &  & 9\\
\hline
\end{tabular}
\end{center}
\newpage

\header{2014--2015}{10 декабря 2014}
\begin{center}
\begin{tabular}{|l|l|l|c|*{10}{p{0.3cm}|}c|c|}
\hline
№ & Участник & Спец & Курс & 1 & 2 & 3 & 4 & 5 & 6 & 7 & 8 & 9 & 10 & $\Sigma$ & Диплом\\
\hline
1 & Амир Мирас &  ВМК & 2 & 0 & 10 & 0 & 9 & 10 & 10 & 0 & 0 & 0 & 0 & 39 & 1\\
\hline
2 & Журавская Александра &  ВМК & 1 & 0 & 0 & 0 & 0 & 10 & 10 & 10 & 0 & 8 & 0 & 38 & 1\\
\hline
3 & Булгаков Анатолий &  ВМК & 2 & 2 & 2 & 0 & 0 & 10 & 9 & 0 & 0 & 0 & 0 & 23 & 3\\
\hline
4 & Шокетаева Надира &  ММ & 2 & 3 & 0 & 0 & 0 & 5 & 10 & 0 & 0 & 0 & 0 & 18 & 3\\
\hline
5 & Абайулы Ерулан &  ВМК & 1 & 0 & 0 & 0 & 0 & 10 & 7 & 0 & 0 & 0 & 0 & 17 & 3\\
\hline
6-8 & Таскынов Ануар &  ВМК & 2 & 0 & 0 & 0 & 0 & 0 & 10 & 0 & 0 & 0 & 0 & 10 & \\
\hline
6-8 & Токтаганов Адильхан &  ММ & 1 & 0 & 0 & 0 & 0 & 0 & 10 & 0 & 0 & 0 & 0 & 10 & \\
\hline
6-8 & Таранов Денис &  ВМ & 2 & 0 & 0 & 0 & 0 & 9 & 0 & 0 & 1 & 0 & 0 & 10 & \\
\hline
9 & Батырбеков Аскар &  ММ & 1 & 0 & 0 & 0 & 0 & 0 & 9 & 0 & 0 & 0 & 0 & 9 & \\
\hline
10 & Кенесова Аида &  ВМ & 1 & 0 & 0 & 0 & 0 & 0 & 2 & 0 & 0 & 0 & 0 & 2 & \\
\hline
11 & Даку Ансар &  ВМ & 1 & 0 & 0 & 0 & 0 & 1 & 0 & 0 & 0 & 0 & 0 & 1 & \\
\hline
\end{tabular}

\end{center}
\newpage

\header{2015--2016}{19 декабря 2015}
\input{results/2015}
\newpage

\header{2016--2017}{10 декабря 2016}
\begin{flushright}
Стоимость задач: \\
10 баллов каждая задача.\\
\end{flushright}

\begin{enumerate}
\item Введём функцию
$$
f(n) = \bigl[\sqrt 1\,\bigr] + \bigl[\sqrt 2\,\bigr] + \bigl[\sqrt 3\,\bigr] + \hdots + \bigl[\sqrt{n^2-1}\,\bigr] + \bigl[\sqrt{n^2}\,\bigr],
$$
где $[x]$ --- наибольшее целое число, не превышающее $x$. Опишите функцию, которая вычисляет $f(n)$ для данного натурального $n$, не используя при этом операцию извлечения корня и вещественную арифметику.

\item На декартовой координатной плоскости нарисованы две полупараболы: график функции $y = x^2$ $(x \geqslant 0)$ и его копия, повёрнутая на прямой угол по часовой стрелке. Эти две кривые отсекают от прямой, параллельной оси ординат, отрезок длины $L$. Обозначим через $S(L)$ --- площадь отсечённой фигуры.

a) Докажите, что $S(L) > 1$ при $L > 2$;

б) Напишите функцию, которая вычисляет $S(L)$ для данного положительного вещественного числа $L$.



\item Найдите все дифференцируемые функции $f\colon \mathbb R\rightarrow\mathbb R$, удовлетворяющие соотношению
$$
f(x-y) + f(x+y) = f'(x^2 + y^2)
$$
для любых $x,y\in\mathbb R$.



\item  Дана функция $f\colon [0, 2n]\rightarrow\mathbb R$. Пусть $f_i = f(i)$ --- значения функции во всех целых $i$ от 0 до $2n$. Дана переменная $S$ вещественного типа с начальным значением 0. За один ход робот может выбрать целое $i$ от 1 до $2n-1$, затем добавить к переменной $S$ или вычесть из нее среднее арифметическое значений функции $f(x)$ в узлах~$i-1$,~$i$,~$i+1$:
$$S := S \pm \frac{f_{i-1}+f_i+f_{i+1}}{3}.$$ 
Может ли робот за конечное число ходов получить в переменной $S$ значение
$$I = \frac{1}{3} \left(f_0 + 4\sum\limits_{k=1}^{n}{f_{2k-1}} + 2\sum\limits_{k=1}^{n-1}{f_{2k}} + f_{2n}\right),$$
которое является приближением интеграла $\displaystyle \int\limits_0^{2n} f(x)\,dx$, если\\
а) $f(0) = f(2n) = 0$;\\
б) $f(0) \ne 0$, $f(2n) \ne 0$?

\item  Дана некоторая условная машина, состоящая из памяти в $n$ бит и указателя, который в каждый отдельный момент находится над какой-то из этих n ячеек. Перед запуском программы в память записывается некоторое натуральное число $m$ в двоичной системе счисления, а указатель устанавливается над крайним правым (младшим) битом числа. Язык программирования для этой машины состоит из следующих команд:

\begin{center}
\begin{tabular}{|c|c|p{14cm}|}
\hline
{\bf L} & left & сместить указатель налево на одну ячейку, если это возможно, иначе завершить программу\\
\hline
{\bf R} & right & сместить указатель направо на одну ячейку, если это возможно, иначе завершить программу\\
\hline
{\bf C} & change & изменить значение бита в текущей ячейке на противоположное\\
\hline
{\bf A} & again & перейти к выполнению первой команды\\
\hline
{\bf S} & skip & пропустить две следующие команды, если в текущей ячейке 0\\
\hline
{\bf F} & finish & завершить выполнение программы\\
\hline
\end{tabular}
\end{center}

Команды записываются в одну строку и выполняются в последовательном порядке, слева направо. При этом запись программы обязана оканчиваться командой \textbf{A} или \textbf{F}. Напишите для этой абстрактной машины следующие программы:

а) заменить данное число на $(m - 1)$;

б) заменить данное число на $(2^n - m - 1)$;

в) изменить на противоположный его старший (крайний слева) бит.

\textit{Примеры:}

а) программа, обнуляющая все ячейки: \textbf{SSCLA};

б) программа, которая изменяет второй справа бит, если крайний справа бит нулевой: \textbf{SFFLCF}.


\item Из квадратной однородной пластины со стороной 1 случайным образом вырезается квадрат со сторонами, равными $2a$ и параллельными сторонам исходного квадрата. При этом центр квадрата --- это случайная величина, равномерно распределённая по всем допустимым положениям (квадрат со стороной $(1 - 2a)$).

a) Найдите вероятность $p(a)$ того, что центр тяжести полученной фигуры лежит в вырезанной области.

б) Опишите функцию $p(a)$, которая вычисляет указанную вероятность приблизительно, считая при этом, что нам не известен метод нахождения центра тяжести произвольной фигуры, однако мы можем найти центр тяжести конечного множества точек одинаковой массы.

\end{enumerate}

\newpage

\header{2017--2018}{9 декабря 2017}
\begin{center}
\begin{tabular}{|l|l|l|c|c|c|c|c|c|c|c|c|}
\hline
№ & Участник & ВУЗ & Курс & 1 & 2 & 3 & 4 & 5 & 6 & $\Sigma$ & Диплом \\
\hline
1 & Жанбырбаев Есеналы & КБТУ & 2 & 10 & 10 & 10 & 3 & 0 & 10 & 43 & 1 степени \\
\hline
2 & Бекмаганбетов Бекарыс & КФ МГУ & 1 & 10 & 10 & 10 & 2 & 10 & 0 & 42 & 2 степени \\
\hline
3 & Сайланбаев Алибек & НУ & 4 & 10 & 10 & 10 & 2 & 0 & 8 & 40 & 2 степени \\
\hline
4 & Аманкелды Акежан & НУ & 4 & 9 & 9 & 10 & 5 & 0 & 0 & 33 & 3 степени \\
\hline
5 & Жанахметов Султан & НУ & 3 & 10 & 10 & 10 & 2 & 0 & 0 & 32 & 3 степени \\
\hline
6 & Шакиев Александр & МУИТ & 2 & 10 & 9 & 10 & 0 & 0 & 0 & 29 & 3 степени \\
\hline
\end{tabular}
\end{center}
\newpage

\header{2017--2018 (дополнительный тур)}{13 марта 2018}
\begin{center}
\begin{tabular}{|l|l|l|c|*{5}{p{0.3cm}|}c|}
\hline
№ & Участник & Факультет & Курс & 1 & 2 & 3 & 4 & 5 & $\Sigma$ \\
\hline
1 & Бекмаганбетов Бекарыс & ММ & 2 & 10 & 10 & 10 & 0 & 10 & 40 \\
\hline
2 & Аскергали Ануар & ВМК & 2 & 0 & 0 & 10 & 9 & 0 & 19 \\
\hline
3 & Дукенбай Аслан & ММ & 1 & 10 & 0 & 0 & 0 & 0 & 10 \\
\hline
4 & Ергалиев Иса  & ВМК & 2 & 10 & 0 & 0 & 0 & 0 & 10 \\
\hline
5 & Сурукпаев Аслан & ММ & 2 & 0 & 10 & 0 & 0 & 0 & 10 \\
\hline
6 & Макатова Батима & ММ & 2 & 9 & 0 & 0 & 0 & 0 & 9 \\
\hline
7 & Вагнер Алан & ВМК & 1 & 0 & 0 & 0 & 9 & 0 & 9 \\
\hline
8 & Джексембаев Руслан & ММ & 1 & 9 & 0 & 0 & 0 & 0 & 9 \\
\hline
9 & Ержанов Жалгас & ВМК & 2 & 0 & 0 & 0 & 0 & 0 & 0 \\
\hline
10 & Кунакбаев Рамазан & ММ & 2 & 0 & 0 & 0 & 0 & 0 & 0 \\
\hline
\end{tabular}
\end{center}
\newpage
%%%%%%%%%%%%%%%%%%%%

\header{Республиканская олимпиада по математике 2016}{01 апреля 2016}
\begin{center}
\begin{tabular}{|l|l|l|c|c|c|c|c|c|c|c|c|}
\hline
№ & Участник & ВУЗ & Курс & 1 & 2 & 3 & 4 & 5 & 6 & $\Sigma$ & Диплом \\
\hline
1 & Журавская Александра & КФ МГУ & 2 & 10 & 9 & 0 & 0 & 9 & 0 & 28 & 1 степени \\
\hline
2 & Аманкелды Акежан & НУ & 3 & 3 & 9 & 2 & 0 & 1 & 7 & 22 & 2 степени \\
\hline
3 & Турганбаев Сатбек & КФ МГУ & 2 & 10 & 7 & 3 & 0 & 0 & 1 & 21 & 2 степени \\
\hline
4 & Полищук Руслан & НУ & 1 & 10 & 7 & 0 & 0 & 0 & 3 & 20 & 3 степени \\
\hline
5 & Жанахметов Султан & НУ & 2 & 10 & 5 & 0 & 0 & 1 & 3 & 19 & 3 степени \\
\hline
6 & Токтаганов Адиль & КФ МГУ & 2 & 8 & 0 & 0 & 10 & 1 & 0 & 19 & 3 степени \\
\hline
\end{tabular}
\end{center}
\newpage

\header{Республиканская олимпиада по МКМ 2016}{01 апреля 2016}
\begin{center}
\begin{tabular}{|l|l|l|c|c|c|c|c|c|c|c|c|}
\hline
№ & Участник & ВУЗ & Курс & 1 & 2 & 3 & 4 & 5 & 6 & $\Sigma$ & Диплом \\
\hline
1 & Батырхан Орынкул & НУ & 3 & 10 & 10 & 1 & 5 & 9 & 6 & 41 & 1 степени \\
\hline
2 & Абайулы Ерулан & КФ МГУ & 2 & 5 & 10 & 2 & 2 & 10 & 0 & 29 & 2 степени \\
\hline
3 & Камалбеков Тимур & КФ МГУ & 2 & 8 & 7 & 1 & 0 & 9 & 2 & 27 & 2 степени \\
\hline
4 & Омаров Темирхан & КФ МГУ & 2 & 4 & 10 & 0 & 2 & 10 & 0 & 22 & 3 степени \\
\hline
5 & Сайланбаев Алибек & НУ & 3 & 7 & 5 & 3 & 1 & 6 & 0 & 22 & 3 степени \\
\hline
6 & Иманмалик Ержан & НУ & 2 & 9 & 3 & 3 & 5 & 1 & 0 & 21 & 3 степени \\
\hline
\end{tabular}
\end{center}

\newpage

\header{Республиканская олимпиада по математике 2017}{13 апреля 2017}
\begin{center}
\begin{tabular}{|l|l|l|c|c|c|c|c|c|c|c|c|}
\hline
№ & Участник & ВУЗ & Курс & 1 & 2 & 3 & 4 & 5 & 6 & $\Sigma$ & Диплом \\
\hline
1 & Жанбырбаев Есеналы & КБТУ & 2 & 10 & 10 & 10 & 3 & 0 & 10 & 43 & 1 степени \\
\hline
2 & Бекмаганбетов Бекарыс & КФ МГУ & 1 & 10 & 10 & 10 & 2 & 10 & 0 & 42 & 2 степени \\
\hline
3 & Сайланбаев Алибек & НУ & 4 & 10 & 10 & 10 & 2 & 0 & 8 & 40 & 2 степени \\
\hline
4 & Аманкелды Акежан & НУ & 4 & 9 & 9 & 10 & 5 & 0 & 0 & 33 & 3 степени \\
\hline
5 & Жанахметов Султан & НУ & 3 & 10 & 10 & 10 & 2 & 0 & 0 & 32 & 3 степени \\
\hline
6 & Шакиев Александр & МУИТ & 2 & 10 & 9 & 10 & 0 & 0 & 0 & 29 & 3 степени \\
\hline
\end{tabular}
\end{center}
\newpage

\end{scriptsize}
\end{landscape}

\end{document}