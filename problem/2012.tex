\begin{enumerate}

\item Будем называть целочисленную квадратную матрицу порядка $n$ <<весёлой>>, если сумма элементов каждой её строки чётна. Докажите, что для любой целочисленной матрицы $A$ и для любой <<весёлой>> матрицы $B$ произведение $AB$ тоже является <<весёлой>> матрицей.

\item  Пусть ($A$, $+$, $*$) --- конечное кольцо с единицей, в котором $1 + 1 = 0$. Докажите, что уравнения
$x^2 = 0$ и $x^2 = 1$ имеют одинаковое количество корней в кольце.

\item Прямые $l_1$ и $l_2$ касаются параболы в точках $A$ и $B$, соответственно, и пересекаются в точке $C$. Точка $K$ --- середина отрезка $AB$. Докажите,что середина отрезка $KC$ лежит на параболе.

\item При каких натуральных $n$ выполняется равенство
$$\sum\limits_{k=1}^n \left[ \sqrt[k]{n}\,\right] = 2 n ?$$

\item Последовательность 1, 3, 4, 9, 10, 12, 13, $\hdots$ состоит из всех натуральных чисел, являющихся степенью 3 или представимых в виде суммы различных степеней 3. Найдите 100-й член этой последовательности.

\item Квадратная матрица $A$ порядка $m$ такова, что $A^{n+1} = 0$, а $A^n$ не равна нулевой. Докажите, что 
$$a_n A^n + a_{n-1} A^{n-1} + \hdots + a_1 A + a_0 E = 0$$
тогда, и только тогда, когда вещественные коэффициенты $a_i = 0$ при любом $i = 0, 1, \hdots, n$. Здесь $E$ --- единичная матрица.

\item Вычислите сумму ряда
$$\sum_{k=1}^{\infty} \frac{1}{1^3 + 2^3 + \hdots + k^3}.$$
Примечание: соотношение $\zeta(2) = \sum\limits_{n=1}^{\infty} \frac{1}{n^2} = \frac{\pi^2}{6}$ считать известным.

\item Докажите, что 11 коней не могут побить все оставшиеся 53 поля шахматной доски.

\item Пусть $f: [0, 1] \rightarrow \mathbb{R}$ --- непрерывная функция, для которой $\int\limits_{0}^{1} x f(x)\,dx = 0$. Докажите, что
$$\int\limits_{0}^{1} f^2(x)\,dx \geqslant 4 \left( \int\limits_{0}^{1} f(x)\,dx \right)^2.$$


\end{enumerate}

