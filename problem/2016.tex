\begin{enumerate}
\item Докажите, что для любого натурального $n$ верно равенство:
$$\int\limits_0^{2\pi} {\sin(\sin x + nx)}~dx = 0.$$

\item Существует ли такой многочлен $P(x)$, что $P(1) = 2$, $P(2) = 1$ и $P(x)$ принимает иррациональные значения для всех рациональных $x$, кроме 1 и 2? 

\item Для положительных чисел $a_i$ известно, что 
$$a_1 \cdot a_2 \cdot \ldots \cdot a_n = 1.$$ 
Докажите, что функция 
$$f(x) = (1 + a_1^x)(1 + a_2^x) \dots (1 + a_n^x)$$
неубывающая при $x > 0$.

\item Дан тетраэдр, грани которого имеют одинаковую площадь. Докажите, что все его грани равны.

\item Для набора вещественных чисел $c_0, c_1, \dots, c_n$ известно, что 
$$c_0 + \frac{c_1}{2} + \frac{c_2}{3} + \ldots + \frac{c_n}{n+1} = 0.$$
Докажите, что уравнение $c_0 + c_1 x + c_2 x^2 + \ldots + c_n x^n = 0$ имеет хотя бы один вещественный корень. 

\item $A$ --- ассоциативное кольцо с единицей (не гарантируется, что умножение коммутативно). $D$ --- множество всех необратимых элементов $A$. Известно, что $a^2 = 0$ для всех элементов $a \in D$. Докажите, что $axa = 0$ для всех $a \in D$, $x \in A$.

\item Дана матрица $A$ размером $n \times n$, где элементы матрицы $a_{ij}$ равны последней цифре числа $(i+j-2)$. 
\begin{enumerate}
\item Вычислите определитель матрицы при $n \leqslant 8$;
\item Вычислите определитель матрицы при $n \geqslant 11$;
\item Докажите, что определитель матрицы делится на $10^7$ при $n = 9$ и $n = 10$.
\end{enumerate}

\item Обозначим частичную сумму гармонического ряда через 
$$H_n = \frac{1}{1} + \frac{1}{2} + \frac{1}{3} + \ldots + \frac{1}{n}.$$ 
Найти сумму следующего ряда:
$$\frac{H_1}{10} + \frac{H_2}{100} + \frac{H_3}{1000} + \ldots$$
\end{enumerate}