\begin{enumerate}

\item Какие натуральные числа представимы в виде $x^2-y^2+2x+2y$ для некоторых целых $x$ и $y$?

\item Пусть  $\alpha(x)$ --- первая цифра после запятой в десятичной записи числа $2^x$.\\
а) Докажите, что функция $\alpha(x)$ интегрируема по Риману на $[0, 1]$.\\
б) Докажите, что $\displaystyle 3.5 < \int\limits_{0}^{1} \alpha(x) dx < 4.5$.

\item В конечном поле произведение всех ненулевых элементов не равно единице. Докажите, что сумма всех элементов поля равна нулю.

\item В эллипсе с фокусами $F_1$ и $F_2$ проведена хорда $MN$, которая проходит через фокус $F_2$. На прямой $F_1F_2$ выбраны две точки $S$ и $T$ такие, что прямые $SM$ и $TN$ являются касательными к эллипсу. Точка $D$ симметрична $F_2$ относительно прямой $SM$, точка $E$ симметрична $F_2$ относительно $NT$. Прямые $DS$, $TE$ и $MN$ при пересечении образуют треугольник $ABC$ (точка $C$ не лежит на $MN$). Докажите, что:
\begin{itemize}
\item[а)] $CF_2$ --- медиана треугольника $ABC$;
\item[б)] $CF_1$ --- биссектриса треугольника $ABC$.
\end{itemize}

\item Максималист и минималист по очереди вписывают по одному числу в таблицу размера $n \times n$ (последовательно, строчка за строчкой, слева направо и сверху вниз). Каким окажется ранг получившейся матрицы, если максималист изо всех сил старается его максимизировать, а минималист --- минимизировать? (Ответ может зависеть от $n$ и от того, кто делает первый ход.) 
 
\item Функция $f : (1, +\infty) \rightarrow \mathbb{R}$ дифференцируема на всей области определения. Известно, что  $$\displaystyle f'(x) = f\left( \frac{x}{x-1} \right) + f(x)$$ для всех $x > 1$ и $\displaystyle \lim\limits_{x \to \infty} \frac{f'(x)}{e^x} = 2$. Докажите, что $f(2) < 20,\hspace{-2pt}16$.

\end{enumerate}
