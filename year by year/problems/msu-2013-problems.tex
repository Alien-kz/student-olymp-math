\documentclass[12pt, a4paper]{article}

\usepackage[T2A]{fontenc}
\usepackage[utf8]{inputenc}
\usepackage[english, russian]{babel}
\usepackage{amssymb}
\usepackage{amsfonts}
\usepackage{amsmath}
\usepackage{mathtext}

\usepackage{comment}
\usepackage{geometry}
\geometry{left=1cm, right=1cm, top=2cm, bottom=2cm}

\usepackage{graphicx}
\usepackage{tikz}

\usepackage{wrapfig}
\usepackage{fancybox,fancyhdr}
\sloppy

\setlength{\headheight}{28pt}

\newcommand{\head}[4]
{
	\fancyhf{}
	\pagestyle{fancy}
	\chead{#3, #4}

	\begin{center}
	\begin{large}
	#1 \\
	\textit{#2} \\
	\end{large}
	\end{center}

}

\begin{document}

\head{Открытая студенческая олимпиада по математике \\ Казахстанского филиала МГУ}{20 декабря 2013}{Казахстанский филиал МГУ имени М. В. Ломоносова}{г. Астана}

\begin{enumerate}
\item Можно ли разрезать квадрат 7$\times$7 на 5 частей так, чтобы из них можно было сложить 3 квадрата попарно различных целых площадей?

\item Известно, что $a$, $b$ и $c$ --- корни уравнения $x^3 + px + q = 0$. Вычислите определитель матрицы:
\begin{equation*}
\begin{pmatrix}
a & b & c\\
c & a & b\\
b & c & a\\
\end{pmatrix}.
\end{equation*}

\item На плоскости дана парабола. Найдите множество точек плоскости, из которых парабола видна под прямым углом (т.е. касательные, проведённые из этой точки, перпендикулярны друг другу).

\item Бинарная операция $*$: $\mathbb{R} \times \mathbb{R} \rightarrow \mathbb{R}$ удовлетворяет соотношению $(a * b) * c = a + b + c$ для любых вещественных чисел $a$, $b$ и $c$. Докажите, что $a * b = a + b$ для любых вещественных $a$ и $b$.

\item Назовем перестановку чисел от $1$ до $N$ <<интересной>>, если никакое число не стоит на своем месте ($a_i \ne i$). Обозначим $S(N)$ количество <<интересных>> перестановок чисел от $1$ до $N$. Вычислите:\\
a) $S(N)$;\\
б) $\lim\limits_{N \rightarrow \infty} \frac{S(N)}{N!}$.

\item Определите все вещественные числа $k$, для которых справедливо соотношение:
$$\int\limits_{1}^{2} \left( 1 + k \ln{x} \right) x^{x^k + k - 1} dx = 15.$$

\item Докажите, что:\\
a) существует бесконечно много целых чисел, не представимых в виде суммы кубов трёх целых чисел (среди которых могут быть равные);\\
б) любое целое число представимо в виде суммы кубов пяти целых чисел (среди которых могут быть равные).

\item Пусть $n$ --- натуральное число, кратное 4. Посчитайте количество различных биекций $f: \{1, \hdots, n\} \rightarrow \{1, \hdots, n\}$ таких, что $f(j) + f^{-1}(j) = n + 1$ для всех $j = 1, \hdots, n$. 

Пример: для биекции $f: (1, 2, 3, 4) \rightarrow (2, 4, 1, 3)$ обратной будет $f^{-1}: (1,2,3,4) \rightarrow (3, 1, 4, 2)$.

\item Известно, что $P(x)$ --- многочлен степени $n$ такой, что для всех $t \in \{ 1, 2, 2^2, \hdots, 2^n \}$ верно $P(t) = \frac{1}{t}$. Найдите $P(0)$.

\item В графе $G$ все вершины степени $k$. При этом в $G$ нет треугольников и для любых двух вершин, у которых нет общего ребра, найдутся ровно две вершины, с которыми есть общие рёбра у каждой из этих двух вершин. Чему равно количество вершин $G$?

\end{enumerate}


\end{document} 
