\documentclass[11pt, a4paper]{article}

\usepackage[T2A]{fontenc}
\usepackage[utf8]{inputenc}
\usepackage[english, russian]{babel}
\usepackage{amssymb}
\usepackage{amsfonts}
\usepackage{amsmath}
\usepackage{mathtext}

\usepackage{comment}
\usepackage{geometry}
\geometry{left=1cm, right=1cm, top=2cm, bottom=2cm}

\usepackage{graphicx}
\usepackage{tikz}

\usepackage{wrapfig}
\usepackage{fancybox,fancyhdr}
\sloppy


\newcommand{\head}[2]
{
	\fancyhf{}
	\pagestyle{fancy}
	\chead{Казахстанский филиал МГУ имени М.В.Ломоносова}

	\begin{center}
	\begin{LARGE}
	#1
	\end{LARGE}
	\end{center}

	\begin{center}
	\begin{large}
	#2
	\end{large}
	\end{center}

	\textit{
	\begin{flushright}
	Время работы:  180 минут\\
	Каждая задача оценивается в 10 баллов.\\
	\end{flushright}
	}

}


\renewcommand\thesubsubsection{}
\renewcommand\thesubsection{}
\renewcommand\thesection{}
\usepackage[indentfirst, explicit]{titlesec}
\titleformat{\section}{\Large\bfseries\center}{}{1em}{#1}
\titleformat{\subsection}{\Large\bfseries\center}{}{1em}{#1}
\titleformat{\subsubsection}{\normalsize\bfseries\center}{}{1em}{#1}

\begin{document}

\head{Открытая олимпиада по математике}{6 декабря 2009}

\begin{enumerate}

\item На доске написано число $-1000000$ (минус миллион). За ход разрешается выбрать какие-нибудь два (или одно) числа и написать их сумму или произведение. Если выбрано одно число, то надо писать или его квадрат, или его удвоенное. Как за 12 таких ходов написать число ноль?

\item Малыш может съесть торт за 10 минут, банку варенья --- за 8 минут и выпить кастрюлю молока --- за 15 минут, а Карлсон может сделать это за 2, 3 и 4 минуты соответственно. За какое наименьшее время они могут вместе покончить с завтраком, состоящим из торта, банки варенья и кастрюли молока?

\item Касательная к гиперболе $y = \displaystyle\frac{1}{x}$ в точке $M$ отсекает от угла $x \geqslant 0$, $y \geqslant 0$ прямоугольный треугольник. Докажите, что:\\
а) точка $M$ --- середина гипотенузы этого треугольника;\\
б) его площадь не зависит от выбора точки $M$.

\item Дифференцируемые функции $f(x)$ и $g(x)$ определены на отрезке $[0, 1]$ таким образом, что:\\
a) $f(0)=f(1)=1$; \\
б) $2001\cdot f'(x) g(x) + 2009\cdot f(x) g'(x) \geqslant 0$ для всех $x \in [0, 1]$.\\
Докажите, что $g(1) \geqslant g(0)$.

\item Найдите сумму ряда
$$\sum_{n=1}^{\infty} \arctg \left( \frac{1}{n^2+n+1} \right).$$

\item Пусть $\{ q_k \}_{k=1}^{n}$ --- конечная последовательность чисел. Известно, что $q_k \geqslant 1$ для всех $k$. Докажите неравенство:
$$ \sqrt[q_1]{1 + \sqrt[q_2]{ 1 + \hdots + \sqrt[q_n]{1}}} \leqslant 1 + \frac{1}{q_1} + \frac{1}{q_1 q_2} + \hdots + \frac{1}{q_1 \hdots q_n}. $$

\item Найдите минимум величины $\max\limits_{1 \leqslant i < j \leqslant n+1} (x_i, x_j)$ по всем наборам единичных векторов $x_1$, $x_2$, ..., $x_{n+1} \in \mathbb{R}^n$.

\end{enumerate}


\end{document} 
