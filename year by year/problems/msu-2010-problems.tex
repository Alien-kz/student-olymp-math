\documentclass[12pt, a4paper]{article}

\usepackage[T2A]{fontenc}
\usepackage[utf8]{inputenc}
\usepackage[english, russian]{babel}
\usepackage{amssymb}
\usepackage{amsfonts}
\usepackage{amsmath}
\usepackage{mathtext}

\usepackage{comment}
\usepackage{geometry}
\geometry{left=1cm, right=1cm, top=2cm, bottom=2cm}

\usepackage{graphicx}
\usepackage{tikz}

\usepackage{wrapfig}
\usepackage{fancybox,fancyhdr}
\sloppy

\setlength{\headheight}{28pt}

\newcommand{\head}[4]
{
	\fancyhf{}
	\pagestyle{fancy}
	\chead{#3, #4}

	\begin{center}
	\begin{large}
	#1 \\
	\textit{#2} \\
	\end{large}
	\end{center}

}

\begin{document}

\head{Открытая студенческая олимпиада по математике \\ Казахстанского филиала МГУ}{12 декабря 2010}{Казахстанский филиал МГУ имени М. В. Ломоносова}{г. Астана}

\begin{enumerate}

\item Пусть функции $f(x)$, $g(x)$ --- непрерывные и сюръективные отображения из $[0, 1]$ в $[0, 1]$. Докажите, что найдётся точка $x_0$ из отрезка $[0, 1]$ такая, что $f(g(x_0)) = g(f(x_0))$.
\begin{comment}
\item На $n$ человек надевают шапки $k$ цветов (шапок каждого цвета может быть любое количество). Люди видят, что надето на других, но свою шапку не видит никто. Затем каждый пишет на бумажке цвет; если указанный человеком цвет совпал с цветом его шапки, то всей команде даётся 1 балл. Сколько баллов может гарантировать себе команда из $n$ человек?
\end{comment}
\item Пусть $r \in\mathbb N$. Найти предел
$$\lim_{n \rightarrow \infty} \frac{1}{n^{r+1}} \sum_{k=1}^n k^r \cos \frac{k}{n}.$$

\item Пусть $\sum\limits_{n=1}^{\infty} a_n = S$. Доказать, что
$$\lim_{x \rightarrow 1 - 0} \left( a_1 + a_2 x + \hdots + a_n x^{n-1} + \hdots \right) = S.$$

\item Пусть $f$ есть (не обязательно дифференцируемая) функция, удовлетворяющая для любой пары $x_1 < x_2$ неравенству
$$ f \left( \frac{x_1 + x_2}{2} \right) < \frac{f(x_1) + f(x_2)}{2}.$$
Доказать, что тогда верно неравенство
$$ f \left( \frac{x_1 + x_2 + \hdots + x_n}{n} \right) < \frac{f(x_1) + f(x_2) + \hdots + f(x_n)}{n}.$$

\item У числа $a$ есть $p$ делителей, $p$ --- простое число. Докажите, что $a (a^k - 1)$ делится на $p$ для любого натурального $k$.

\item Существуют ли квадратные матрицы $A$, $B$ такие, что $AB - BA = E$, где $E$ --- единичная матрица?

\item Рассмотрим ряд $\sum\limits_{n=1}^{\infty} \frac{\cos{n}}{n}$. Разобьем $N$-ю частичную сумму этого ряда на два слагаемых:
$$ S_N = \sum_{n=1}^{N} \frac{\cos{n}}{n} = S_N^{+} + S_N^{-},$$
где $S_N^{+}$ и $S_N^{-}$ --- суммы соответственно положительных и отрицательных членов. Докажите, что существует предел $\lim\limits_{N \rightarrow \infty}{\frac{S_N^{+}}{S_N^{-}}}$ и найдите его.

\item Дано конечное множество точек $\Delta = \{ A_1, A_2, \hdots, A_n \}$ на плоскости и положительное число $\rho > 0$. Для произвольной точки $X$ плоскости построим последовательность точек $\{X_k\}_{k=1}^{\infty}$ по следующему правилу: $X_1 = X$, $X_{k+1}$ --- это центр тяжести точек из $\Delta$, содержащихся в круге радиуса $\rho$ с центром в точке $X_k$, если такие точки существуют, и $X_{k+1} = X_k$ иначе. Докажите, что при любом выборе начальной точки $X$ данная последовательность будет постоянной, начиная с некоторого номера.

\end{enumerate}

\end{document} 
