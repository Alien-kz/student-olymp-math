\documentclass[11pt, a4paper]{article}

\usepackage[T2A]{fontenc}
\usepackage[utf8]{inputenc}
\usepackage[english, russian]{babel}
\usepackage{amssymb}
\usepackage{amsfonts}
\usepackage{amsmath}
\usepackage{mathtext}
\usepackage{comment}
\usepackage{graphicx}

\usepackage{tikz}
\usepackage{geometry}
\geometry{left=1cm, right=1cm, top=2cm, bottom=2cm}
\sloppy
\usepackage{wrapfig}
\graphicspath{{../../}}
\usepackage{fancybox,fancyhdr}
\newcommand{\informat}[1]
{
	\paragraph{Ввод.\\} #1
}

\newcommand{\outformat}[1]
{
	\paragraph{Вывод.\\} #1
}

\newcommand{\example}[2]
{
	\paragraph{Пример.\\}
	{\tt
	\begin{tabular}{|p{0.45\linewidth}|p{0.45\linewidth}|}
	\hline
	Ввод & Вывод \\
	\hline
	#1 & #2		\\
	\hline
	\end{tabular}
	}
}

\newcommand{\examplee}[4]
{
	\paragraph{Пример.\\}
	{\tt
	\begin{tabular}{|p{0.45\linewidth}|p{0.45\linewidth}|}
	\hline
	Ввод 	& Вывод  	\\
	\hline
	#1 		& #2 		\\
	\hline
	#3		& #4		\\
	\hline
	\end{tabular}
	}
}

\newcommand{\exampleee}[6]
{
	\paragraph{Пример.\\}
	{\tt
	\begin{tabular}{|p{0.45\linewidth}|p{0.45\linewidth}|}
	\hline
	Ввод 	& Вывод  	\\
	\hline
	#1 		& #2 		\\
	\hline
	#3		& #4		\\
	\hline
	#5		& #6		\\
	\hline
	\end{tabular}
	}
}

\newcommand{\exampleeee}[8]
{
	\paragraph{Пример.\\}
	{\tt
	\begin{tabular}{|p{0.45\linewidth}|p{0.45\linewidth}|}
	\hline
	Ввод 	& Вывод  	\\
	\hline
	#1 		& #2 		\\
	\hline
	#3		& #4		\\
	\hline
	#5		& #6		\\
	\hline
	#7		& #8		\\
	\hline
	\end{tabular}
	}
}

\newcommand{\exampleeeee}[5]
{
	\paragraph{Пример.\\}
	{\tt
	\begin{tabular}{|p{0.45\linewidth}|p{0.45\linewidth}|}
	\hline
	Ввод 	& Вывод  	\\
	\hline
	#1		\\
	\hline
	#2		\\
	\hline
	#3		\\
	\hline
	#4		\\
	\hline
	#5		\\
	\hline
	\end{tabular}
	}
}


\newcommand{\excomm}[1]
{
	\paragraph{Комментарий. \\}
	\textit{#1}
}

\newcommand{\head}[2]
{
	\fancyhf{}
	\pagestyle{fancy}
%	\chead{Олимпиада }
%	\rhead{#2}

	\chead{Казахстанский филиал МГУ имени М.В.Ломоносова}

\begin{center}
\begin{LARGE}
#1
\end{LARGE}
\end{center}

\begin{center}
\begin{large}
#2
\end{large}
\end{center}

\textit{
\begin{flushright}
Время работы:  180 минут\\
Каждая задача оценивается в 10 баллов.\\
\end{flushright}
}

}


\renewcommand\thesubsubsection{}
\renewcommand\thesubsection{}
\renewcommand\thesection{}
\usepackage[indentfirst, explicit]{titlesec}
\titleformat{\section}{\Large\bfseries\center}{}{1em}{#1}
\titleformat{\subsection}{\Large\bfseries\center}{}{1em}{#1}
\titleformat{\subsubsection}{\normalsize\bfseries\center}{}{1em}{#1}

\begin{document}

\head{Открытая олимпиада по математике}{10 декабря 2016}
\begin{enumerate}
\item Докажите, что для любого натурального $n$ верно равенство:
$$\int\limits_0^{2\pi} {\sin(\sin x + nx)}~dx = 0.$$

\item Существует ли такой многочлен $P(x)$, что $P(1) = 2$, $P(2) = 1$ и $P(x)$ принимает иррациональные значения для всех рациональных $x$, кроме 1 и 2? 

\item Для положительных чисел $a_i$ известно, что 
$$a_1 \cdot a_2 \cdot \ldots \cdot a_n = 1.$$ 
Докажите, что функция 
$$f(x) = (1 + a_1^x)(1 + a_2^x) \dots (1 + a_n^x)$$
неубывающая при $x > 0$.

\item Дан тетраэдр, грани которого имеют одинаковую площадь. Докажите, что все его грани равны.

\item Для набора вещественных чисел $c_0, c_1, \dots, c_n$ известно, что 
$$c_0 + \frac{c_1}{2} + \frac{c_2}{3} + \ldots + \frac{c_n}{n+1} = 0.$$
Докажите, что уравнение $c_0 + c_1 x + c_2 x^2 + \ldots + c_n x^n = 0$ имеет хотя бы один вещественный корень. 

\item $A$ --- ассоциативное кольцо с единицей (не гарантируется, что умножение коммутативно). $D$ --- множество всех необратимых элементов $A$. Известно, что $a^2 = 0$ для всех элементов $a \in D$. Докажите, что $axa = 0$ для всех $a \in D$, $x \in A$.

\item Дана матрица $A$ размером $n \times n$, где элементы матрицы $a_{ij}$ равны последней цифре числа $(i+j-2)$. 
\begin{enumerate}
\item Вычислите определитель матрицы при $n \leqslant 8$;
\item Вычислите определитель матрицы при $n \geqslant 11$;
\item Докажите, что определитель матрицы делится на $10^7$ при $n = 9$ и $n = 10$.
\end{enumerate}

\item Обозначим частичную сумму гармонического ряда через 
$$H_n = \frac{1}{1} + \frac{1}{2} + \frac{1}{3} + \ldots + \frac{1}{n}.$$ 
Найти сумму следующего ряда:
$$\frac{H_1}{10} + \frac{H_2}{100} + \frac{H_3}{1000} + \ldots$$
\end{enumerate}

\end{document} 
