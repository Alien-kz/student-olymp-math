\documentclass[11pt, a4paper]{article}

\usepackage[T2A]{fontenc}
\usepackage[utf8]{inputenc}
\usepackage[english, russian]{babel}
\usepackage{amssymb}
\usepackage{amsfonts}
\usepackage{amsmath}
\usepackage{mathtext}
\usepackage{comment}
\usepackage{graphicx}

\usepackage{tikz}
\usepackage{geometry}
\geometry{left=1cm, right=1cm, top=2cm, bottom=2cm}
\sloppy
\usepackage{wrapfig}
\graphicspath{{../../}}
\usepackage{fancybox,fancyhdr}
\newcommand{\informat}[1]
{
	\paragraph{Ввод.\\} #1
}

\newcommand{\outformat}[1]
{
	\paragraph{Вывод.\\} #1
}

\newcommand{\example}[2]
{
	\paragraph{Пример.\\}
	{\tt
	\begin{tabular}{|p{0.45\linewidth}|p{0.45\linewidth}|}
	\hline
	Ввод & Вывод \\
	\hline
	#1 & #2		\\
	\hline
	\end{tabular}
	}
}

\newcommand{\examplee}[4]
{
	\paragraph{Пример.\\}
	{\tt
	\begin{tabular}{|p{0.45\linewidth}|p{0.45\linewidth}|}
	\hline
	Ввод 	& Вывод  	\\
	\hline
	#1 		& #2 		\\
	\hline
	#3		& #4		\\
	\hline
	\end{tabular}
	}
}

\newcommand{\exampleee}[6]
{
	\paragraph{Пример.\\}
	{\tt
	\begin{tabular}{|p{0.45\linewidth}|p{0.45\linewidth}|}
	\hline
	Ввод 	& Вывод  	\\
	\hline
	#1 		& #2 		\\
	\hline
	#3		& #4		\\
	\hline
	#5		& #6		\\
	\hline
	\end{tabular}
	}
}

\newcommand{\exampleeee}[8]
{
	\paragraph{Пример.\\}
	{\tt
	\begin{tabular}{|p{0.45\linewidth}|p{0.45\linewidth}|}
	\hline
	Ввод 	& Вывод  	\\
	\hline
	#1 		& #2 		\\
	\hline
	#3		& #4		\\
	\hline
	#5		& #6		\\
	\hline
	#7		& #8		\\
	\hline
	\end{tabular}
	}
}

\newcommand{\exampleeeee}[5]
{
	\paragraph{Пример.\\}
	{\tt
	\begin{tabular}{|p{0.45\linewidth}|p{0.45\linewidth}|}
	\hline
	Ввод 	& Вывод  	\\
	\hline
	#1		\\
	\hline
	#2		\\
	\hline
	#3		\\
	\hline
	#4		\\
	\hline
	#5		\\
	\hline
	\end{tabular}
	}
}


\newcommand{\excomm}[1]
{
	\paragraph{Комментарий. \\}
	\textit{#1}
}

\newcommand{\head}[2]
{
	\fancyhf{}
	\pagestyle{fancy}
%	\chead{Олимпиада }
%	\rhead{#2}

	\chead{Казахстанский филиал МГУ имени М.В.Ломоносова}

\begin{center}
\begin{LARGE}
#1
\end{LARGE}
\end{center}

\begin{center}
\begin{large}
#2
\end{large}
\end{center}

\textit{
\begin{flushright}
Время работы:  180 минут\\
Каждая задача оценивается в 10 баллов.\\
\end{flushright}
}

}


\renewcommand\thesubsubsection{}
\renewcommand\thesubsection{}
\renewcommand\thesection{}
\usepackage[indentfirst, explicit]{titlesec}
\titleformat{\section}{\Large\bfseries\center}{}{1em}{#1}
\titleformat{\subsection}{\Large\bfseries\center}{}{1em}{#1}
\titleformat{\subsubsection}{\normalsize\bfseries\center}{}{1em}{#1}

\begin{document}

\head{Открытая олимпиада по математике}{9 декабря 2017}
\begin{enumerate}
\item Привести пример вещественной матрицы $A$ такой, что $A^4 = I$, но при этом $A^2 \neq \pm I$, где $I$ --- единичная матрица.

\item Существует ли такая расходящаяся числовая последовательность $\{x_n\}$, что при любом натуральном $k > 1$ её подпоследовательность $\{x_{kn}\}$ сходится?

\item Вычислите
$$
\int\limits_{-1}^{1} \frac{x^{2k} + 2017}{2018^x + 1} \;dx ,
$$
где $k\in\mathbb Z$.

\item Найдите все непрерывные функции $f: \mathbb{R} \to \mathbb{R}$ такие, что 
$$f(x) + f\left(3 - \frac{9}{x}\right) = x - \frac{9}{x}$$
для всех $x\in\mathbb R\setminus\{0, 3\}.$

\item К параболе проведены две касательные $l_a$ и $l_b$ в точках $A$ и $B$. Точка $C$ симметрична точке $A$ относительно $l_b$, а точка $D$ симметрична точке $B$ относительно $l_a$. Докажите, что точки $A$, $B$, $C$ и $D$ образуют ромб тогда и только тогда, когда $AB$ проходит через фокус параболы.

\textit{Можно использовать оптическое свойство параболы без доказательства.}

\item Найдите все такие натуральные $n$, при которых 
$$C_{n}^0 \cdot C_{n}^1 \cdot \hdots \cdot C_n^n$$
является точным квадратом.

\textit{Здесь $C_n^k = \dfrac{n!}{k!(n-k)!}$ --- биномиальный коэффициент.}

\item Докажите сходимость последовательности $\{a_n\}$, где 
$$
\begin{cases}
a_1 = 1,\\
a_{n+1} = a_n + \sin{a_n},\\
\end{cases}
$$
и найдите её предел.

\item Функция $F(n, k)$, определённая для всех целых неотрицательных $n$ и $k$, удовлетворяет условиям
$$
\begin{cases}
F(n, k) = F(n - 1, k) + F(n, k - 1), & n \geqslant 1, k \geqslant 1,\\
F(n, 0) = 1,  & n \geqslant 0,\\
F(0, k) = 2 F(0, k - 1) + 1, & k \geqslant 1.
\end{cases}
$$
Вычислите $F(n, n)$.
\end{enumerate}

\end{document} 
