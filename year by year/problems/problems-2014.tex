\documentclass[11pt, a4paper]{article}

\usepackage[T2A]{fontenc}
\usepackage[utf8]{inputenc}
\usepackage[english, russian]{babel}
\usepackage{amssymb}
\usepackage{amsfonts}
\usepackage{amsmath}
\usepackage{mathtext}

\usepackage{comment}
\usepackage{geometry}
\geometry{left=1cm, right=1cm, top=2cm, bottom=2cm}

\usepackage{graphicx}
\usepackage{tikz}

\usepackage{wrapfig}
\usepackage{fancybox,fancyhdr}
\sloppy


\newcommand{\head}[2]
{
	\fancyhf{}
	\pagestyle{fancy}
	\chead{Казахстанский филиал МГУ имени М.В.Ломоносова}

	\begin{center}
	\begin{LARGE}
	#1
	\end{LARGE}
	\end{center}

	\begin{center}
	\begin{large}
	#2
	\end{large}
	\end{center}

	\textit{
	\begin{flushright}
	Время работы:  180 минут\\
	Каждая задача оценивается в 10 баллов.\\
	\end{flushright}
	}

}


\renewcommand\thesubsubsection{}
\renewcommand\thesubsection{}
\renewcommand\thesection{}
\usepackage[indentfirst, explicit]{titlesec}
\titleformat{\section}{\Large\bfseries\center}{}{1em}{#1}
\titleformat{\subsection}{\Large\bfseries\center}{}{1em}{#1}
\titleformat{\subsubsection}{\normalsize\bfseries\center}{}{1em}{#1}

\begin{document}

\head{Открытая олимпиада по математике}{10 декабря 2014}

\begin{enumerate}
\item Найдите все непрерывные функции $f: \mathbb{R} \rightarrow \mathbb{R}$, удовлетворяющие уравнению

$$ 3 f(2 x + 1) = f( x ) + 5 x.$$

\item Известно, что

$$ \int\limits_{0}^{1} \frac{\ln(1+x)}{x}\,dx = \frac{\pi^2}{12}.$$

Вычислите

$$ \int\limits_{0}^{1} \frac{\ln(1-x^3)}{x}\,dx.$$

\item Пусть $\{ \varepsilon_n \}_{n=1}^{\infty}$ --- последовательность, состоящая из чисел $1$ и $-1$. Может ли число

$$\sum_{n=1}^{\infty} \frac{\varepsilon_n}{n!}$$

быть рациональным?

\item Для квадратных матриц одного порядка $A$ и $B$ верно $AB = A + 2014B$. Докажите, что сумма коэффициентов характеристического многочлена матрицы $B$ не равна нулю.

\item Двое играют в игру. На доске написано число $n$. За один ход разрешается уменьшить число на любой из его целых положительных делителей (в том числе на единицу или на само это число). Если при этом получается нуль, игрок проиграл. Ходы делаются поочерёдно. Какой из двух игроков выиграет, если они оба играют оптимально?

\item Дано вещественное число $x$ такое, что $x^3 = x + 1$. Доказать, что $x^5 = x^4 + 1$.

\item  Найти сумму ряда

$$\sum_{n=1}^{\infty} {\mathrm{arcctg}}~{F_{2n+1}}, $$
где $F_n$ --- числа Фибоначчи ($F_1=F_2=1$, $F_{n+2} = F_{n+1}+F_n$ для $n\in\mathbb N$).

\item Пусть $a_n$ --- число обратных самих себе перестановок порядка $n$. Доказать $a_{n+1} = a_n + n a_{n-1}$.

\item Прямоугольник размером $N \times M$ ($N>2$, $M>2$) разбит равномерной сеткой на $N M$ квадратов $1 \times 1$. Назовем его красивым, если все клетки $1 \times 1$ можно покрасить в один из двух цветов так, чтобы никакие 4 клетки, образующие прямоугольник со сторонами, параллельными сторонам исходного прямоугольника, не были покрашены в один цвет. Найдите красивый прямоугольник максимальной площади.

\item Докажите, что ортоцентр треугольника, описанного около параболы, лежит на её директрисе.
 
\end{enumerate}

\end{document} 
