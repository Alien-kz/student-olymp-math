\documentclass[11pt, a4paper]{article}

\usepackage[T2A]{fontenc}
\usepackage[utf8]{inputenc}
\usepackage[english, russian]{babel}
\usepackage{amssymb}
\usepackage{amsfonts}
\usepackage{amsmath}
\usepackage{mathtext}
\usepackage{comment}
\usepackage{graphicx}

\usepackage{tikz}
\usepackage{geometry}
\geometry{left=1cm, right=1cm, top=2cm, bottom=2cm}
\sloppy
\usepackage{wrapfig}
\graphicspath{{../../}}
\usepackage{fancybox,fancyhdr}

\newcommand{\head}[2]
{
	\fancyhf{}
	\pagestyle{fancy}
	\chead{Казахстанский филиал МГУ имени М.В.Ломоносова}

	\begin{center}
	\begin{LARGE}
	#1
	\end{LARGE}
	\end{center}

	\begin{center}
	\begin{large}
	#2
	\end{large}
	\end{center}

	\begin{center}
	\begin{LARGE}
	Указания
	\end{LARGE}
	\end{center}
}

\begin{document}

\head{Открытая олимпиада по математике}{10 декабря 2016}

\begin{enumerate}
\item Пусть $I$ --- искомый интеграл. Сделав замену $t=2\pi-x$, получим $I=-I$; следовательно, $I=0$.

\item Так как $P(1) = 2$ и $P(2) = 1$, то многочлен $P(x)$ можно представить в виде $P(x)=(x-1)(x-2)Q(x)+3-x$ для некоторого многочлена $Q(x)$. Осталось подобрать многочлен $Q(x)$ так, чтобы для любого рационального $x$ (кроме, быть может, 1 и 2) $Q(x)$ было иррациональным. Например, подойдёт $Q(x)=\sqrt 2$. Нетрудно показать, что полученный многочлен $P(x)=\sqrt 2(x-1)(x-2)+3-x$ удовлетворяет всем указанным требованиям.

\item Заметим, что $f(x) > 0$, для всех $x \in \mathbb{R}$ и $f(-x) = f(x)$, т.е. $f(x)$ --- четная. Следовательно,
$$f(x) = \sqrt{f(x) f(-x)} = \sqrt{\prod_{i=1}^{n}(a_i^x + a_i^{-x} + 2)},$$  
при этом все множители в произведении положительны и нестрого возрастают при $x >0$.

\item Обозначим вершины $A_1$, $A_2$, $A_3$, $A_4$, а косинус двугранного угла при ребре $A_iA_j$ через $c_{ij}$. Суммарная площадь проекций трёх граней на четвёртую равна площади этой грани:
$$c_{ij} + c_{jk} + c_{ki} = 1,$$
для всех 4 возможных троек $(i, j, k)$. Решая систему из 4 уравнений с 6 неизвестными получим: $c_{12} = c_{34}$, $c_{13} = c_{24}$ и $c_{23} = c_{14}$. Также известно, что высоты тетраэдра на все 4 грани равны (например, из $A_3$ и $A_4$). Значит, равны высоты в гранях (например, из $A_3$ на $A_1A_2$ и из $A_1$ на $A_3A_4$). Это приводит к тому, что противоположные ребра равны.


\item Введём функцию
$$
f(x)=c_0x + \frac{c_1}{2}x^2 + \frac{c_2}{3}x^3 + \hdots + \frac{c_n}{n + 1}x^{n+1}.
$$
Очевидно, $f(0)=0$; из условия следует $f(1)=0$. Тогда по теореме Ролля найдётся хотя бы одно такое число $\xi\in[0,1]$, что $f'(\xi)=0$, а данное уравнение как раз можно переписать в виде $f'(x)=0$. 

\item Пусть $c \in D$ и $v \in A$. Тогда $cv \in D$ и $vc \in D$. От противного. Пусть $ cvw = 1$ для некоторого $w \in A$. Тогда $c = c^2 v w = 0$.

Пусть $a \in D$ и $x \not \in D$. Тогда $axax = 0$, откуда $a x a = 0 \cdot x^{-1} = 0$.

Пусть $c, d \in D$. Тогда $(c + d)^4 = (cd + dc)^2 = 0$. Следовательно, $c + d \in D$, откуда $cd = -dc$.

Если $a \in D$ и $x \in D$, то $axa = a \cdot (-a x) = - a^2 x = 0$.

\item (Баев А.Ж.) 
\begin{enumerate}
\item При $n$ от 3 до 8, вторая строка является полусуммой первой и третьей, следовательно, определитель 0.
\item При $n \geqslant 11$ первая и одиннадцатая строка совпадают.
\item Вычтем из $i$-й строки $(i-1)$-ю для всех $i$ с $n$ строки до $2$. Далее еще раз вычтем из $i$-й строки $(i-1)$-ю для всех $i$ с $n$ строки до $3$. Получится матрица с блоком размера $7 \times 7$, где на побочной диагонали стоят числа $-10$, а остальные элементы нули.
\end{enumerate}

\item В полученном двойном ряде можно поменять местами знаки суммирования:
$$
\sum\limits_{i=1}^{\infty}{\frac{H_i}{10^i}}=
\sum\limits_{i=1}^{\infty}{\left(\frac{1}{10^i}\cdot\sum\limits_{j=1}^{i}{\frac{1}{j}}\right)}=$$
$$=\sum\limits_{j=1}^{\infty}{\left(\frac{1}{j}\cdot\sum\limits_{i=j}^{\infty}{\frac{1}{10^i}}\right)}=
\frac{1}{9}\sum\limits_{j=1}^{\infty}{\frac{1}{j\cdot 10^{j-1}}}.
$$
Искомая сумма равна $\displaystyle\frac{10}{9}S\left(\frac{1}{10}\right)$, где $S(x)=\displaystyle\sum\limits_{n=1}^{\infty}{\frac{x^n}{n}}$. Из равномерной сходимости на множестве $[0, 1/2]$ функционального ряда $S(x)$ и ряда, полученного из него почленным дифференцированием, следует
$$
S'(x)=\sum\limits_{n=1}^{\infty}{x^{n-1}}=\frac{1}{1-x}.
$$
Так как $S(0)=0$, то $S(x)=-\ln(1-x)$. Таким образом,
$$
\frac{H_1}{10}+\frac{H_2}{100}+\frac{H_3}{1000}+ \hdots=\frac{10}{9}\ln{\frac{10}{9}}.$$
\end{enumerate}

\end{document}