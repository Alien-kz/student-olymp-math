\documentclass[11pt, a4paper]{article}

\usepackage[T2A]{fontenc}		%cyrillic output
\usepackage[utf8]{inputenc}		%cyrillic output
\usepackage[english, russian]{babel}	%word wrap
\usepackage{amssymb, amsfonts, amsmath}	%math symbols
\usepackage{mathtext}			%text in formulas
\usepackage{geometry}			%paper format attributes
\usepackage{fancyhdr}			%header
\usepackage{graphicx}			%input pictures
\usepackage{tikz}				%draw pictures
\usetikzlibrary{patterns}		%draw pictures: fill
\usepackage{enumitem}			%enumarate parameters

\geometry{left=1cm, right=1cm, top=2cm, bottom=1cm, headheight=15pt}
\setlist[enumerate]{leftmargin=*}	%remove enumarate indenttion
\sloppy							%correct overfull

\newcommand{\head}[4]
{
	\pagestyle{fancy}
	\fancyhf{}
	\chead{#3, #4}

	\begin{center}
	\begin{large}
	#1 \\
	\textit{#2}\\
	\end{large}
	\end{center}

}

\begin{document}

\head{Открытая студенческая олимпиада по математике \\ Казахстанского филиала МГУ}{10 декабря 2014}{Казахстанский филиал МГУ имени М. В. Ломоносова}{г. Астана}

\begin{center}
\begin{tabular}{|l|l|l|c|*{10}{p{0.3cm}|}c|c|}
\hline
№ & Участник & Спец & Курс & 1 & 2 & 3 & 4 & 5 & 6 & 7 & 8 & 9 & 10 & $\Sigma$ & Диплом\\
\hline
1 & Амир Мирас &  ВМК & 2 & 0 & 10 & 0 & 9 & 10 & 10 & 0 & 0 & 0 & 0 & 39 & 1\\
\hline
2 & Журавская Александра &  ВМК & 1 & 0 & 0 & 0 & 0 & 10 & 10 & 10 & 0 & 8 & 0 & 38 & 1\\
\hline
3 & Булгаков Анатолий &  ВМК & 2 & 2 & 2 & 0 & 0 & 10 & 9 & 0 & 0 & 0 & 0 & 23 & 3\\
\hline
4 & Шокетаева Надира &  ММ & 2 & 3 & 0 & 0 & 0 & 5 & 10 & 0 & 0 & 0 & 0 & 18 & 3\\
\hline
5 & Абайулы Ерулан &  ВМК & 1 & 0 & 0 & 0 & 0 & 10 & 7 & 0 & 0 & 0 & 0 & 17 & 3\\
\hline
6-8 & Таскынов Ануар &  ВМК & 2 & 0 & 0 & 0 & 0 & 0 & 10 & 0 & 0 & 0 & 0 & 10 & \\
\hline
6-8 & Токтаганов Адильхан &  ММ & 1 & 0 & 0 & 0 & 0 & 0 & 10 & 0 & 0 & 0 & 0 & 10 & \\
\hline
6-8 & Таранов Денис &  ВМ & 2 & 0 & 0 & 0 & 0 & 9 & 0 & 0 & 1 & 0 & 0 & 10 & \\
\hline
9 & Батырбеков Аскар &  ММ & 1 & 0 & 0 & 0 & 0 & 0 & 9 & 0 & 0 & 0 & 0 & 9 & \\
\hline
10 & Кенесова Аида &  ВМ & 1 & 0 & 0 & 0 & 0 & 0 & 2 & 0 & 0 & 0 & 0 & 2 & \\
\hline
11 & Даку Ансар &  ВМ & 1 & 0 & 0 & 0 & 0 & 1 & 0 & 0 & 0 & 0 & 0 & 1 & \\
\hline
\end{tabular}

\end{center}

\end{document} 
