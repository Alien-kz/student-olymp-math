\documentclass[11pt, a4paper]{article}

\usepackage[T2A]{fontenc}
\usepackage[utf8]{inputenc}
\usepackage[english, russian]{babel}
\usepackage{amssymb}
\usepackage{amsfonts}
\usepackage{amsmath}
\usepackage{mathtext}

\usepackage{comment}
\usepackage{geometry}
\geometry{left=0.5cm, right=1cm, top=1cm, bottom=1cm}
\usepackage[inline]{enumitem}

\usepackage{graphicx}
\usepackage{tikz}
\usetikzlibrary{patterns}

\usepackage{wrapfig}
\usepackage{fancybox,fancyhdr}
\sloppy

\setlength{\headheight}{28pt}
\newcommand{\variant}[2]{
	\begin{center}
	\textit{Вариант #2}
	\end{center}
}

\newcommand{\unit}[1]{\text{\textit{ #1}}}
\newcommand{\units}[2]{ \frac{\text{\textit{#1}}}{\text{\textit{#2}}}}

\newcommand{\head}[4]
{
	\fancyhf{}
	\pagestyle{fancy}
	\chead{#3, #4}

	\begin{center}
	\begin{large}
	#1 \\
	\textit{#2}\\
	\end{large}
	\end{center}

}

\begin{document}

\head{Открытая студенческая олимпиада по математике \\ Казахстанского филиала МГУ}{10 декабря 2011}{Казахстанский филиал МГУ имени М. В. Ломоносова}{г. Астана}

\begin{enumerate}
\item Назовем конечное числовое множество своеобразным, если оно содержит число, равное количеству его элементов, но никакое его собственное подмножество этим свойством не обладает. Определите количество своеобразных подмножеств множества $\{1,\;2,\;\dots,\;12\}$.

\item Определите множество значений функции, сопоставляющей каждому прямоугольному треугольнику отношение~$\displaystyle\frac{h}{r}$, где $h$ --- высота, проведенная к гипотенузе, а $r$~---~радиус вписанной в треугольник окружности.

\item Две последовательности $\{x_n\}_{n=0}^{\infty}$ и $\{y_n\}_{n=0}^\infty$ удовлетворяют условиям:
$$
\begin{cases}
x_{n+1}=2x_n-\alpha y_n,\\
y_{n+1}=2y_n-\displaystyle\frac{1}{\alpha} x_n
\end{cases}
$$
при всех $n\geqslant 0$, где $\alpha\not=0$ --- постоянная величина, а $x_0=1$ и $y_0=0$. Найдите $x_{2012}$ и $y_{2012}$.

\item Существует ли такая последовательность вещественных чисел $\{x_n\}_{n=1}^{\infty}$, что для неё справедливы соотношения: \begin{equation*}
\left\{
\begin{aligned}
\lim\limits_{k\to\infty}x_{12k} = 20,\\
\lim\limits_{k\to\infty}x_{20k} = 12?\\
\end{aligned}
\right.
\end{equation*}

\item Найдите $\displaystyle\int\limits_{0}^{\pi/2}\left(\sin^2(\sin^2x)+\cos^2(\cos^2x)\right)\,dx$.

\item Найдите $\displaystyle\sum\limits_{n=0}^{\infty}\frac{n+2}{n!+(n+1)!+(n+2)!}$.

\item Найдите все функции $f\colon[0,\;1]\to \mathbb R$, удовлетворяющие неравенству $(x-y)^2\leqslant |f(x)-f(y)|\leqslant|x-y|$ для любых $x, y\in[0,\;1]$.

\item Пусть $x_1,\;x_2,\;\dots,\;x_n,\;\dots$ --- все положительные корни уравнения $\mathrm{tg}\,x=x$, выписанные в порядке возрастания, $n_1,\;n_2,\;\dots,\;n_k,\;\dots$ --- некоторая возрастающая последовательность натуральных чисел. Докажите, что ряды $\displaystyle\sum\limits_{k=1}^{\infty} |\cos x_{n_k}|$ и $\displaystyle\sum\limits_{k=1}^{\infty}\frac{1}{n_k}$ сходятся или расходятся одновременно.

\item Квадратная матрица $A$ порядка $n$ состоит из чисел $+1$ и $-1$. При этом, в каждой строке и в каждом столбце этой матрицы находится ровно одно число $-1$. Найдите $|\det A|$.

\item Пусть $S$ --- сумма всех обратимых элементов конечного ассоциативного кольца с единицей. Докажите, что $S^2=0$ или $S^2=S$.
\end{enumerate}


\end{document} 
