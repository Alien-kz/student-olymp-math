\documentclass[11pt, a4paper]{article}

\usepackage[T2A]{fontenc}
\usepackage[utf8]{inputenc}
\usepackage[english, russian]{babel}
\usepackage{amssymb}
\usepackage{amsfonts}
\usepackage{amsmath}
\usepackage{mathtext}

\usepackage{comment}
\usepackage{geometry}
\geometry{left=0.5cm, right=1cm, top=1cm, bottom=1cm}
\usepackage[inline]{enumitem}

\usepackage{graphicx}
\usepackage{tikz}
\usetikzlibrary{patterns}

\usepackage{wrapfig}
\usepackage{fancybox,fancyhdr}
\sloppy

\setlength{\headheight}{28pt}
\newcommand{\variant}[2]{
	\begin{center}
	\textit{Вариант #2}
	\end{center}
}

\newcommand{\unit}[1]{\text{\textit{ #1}}}
\newcommand{\units}[2]{ \frac{\text{\textit{#1}}}{\text{\textit{#2}}}}

\newcommand{\head}[4]
{
	\fancyhf{}
	\pagestyle{fancy}
	\chead{#3, #4}

	\begin{center}
	\begin{large}
	#1 \\
	\textit{#2}\\
	\end{large}
	\end{center}

}

\begin{document}

\head{Открытая студенческая олимпиада по математике \\ Казахстанского филиала МГУ}{19 декабря 2017}{Казахстанский филиал МГУ имени М. В. Ломоносова}{г. Астана}

\begin{enumerate}
\item Привести пример вещественной матрицы $A$ такой, что $A^4 = I$, но при этом $A^2 \neq \pm I$, где $I$ --- единичная матрица.

\item Существует ли такая расходящаяся числовая последовательность $\{x_n\}$, что при любом натуральном $k > 1$ её подпоследовательность $\{x_{kn}\}$ сходится?

\item Вычислите
$$
\int\limits_{-1}^{1} \frac{x^{2k} + 2017}{2018^x + 1} \;dx ,
$$
где $k\in\mathbb Z$.

\item Найдите все непрерывные функции $f: \mathbb{R} \to \mathbb{R}$ такие, что 
$$f(x) + f\left(3 - \frac{9}{x}\right) = x - \frac{9}{x}$$
для всех $x\in\mathbb R\setminus\{0, 3\}.$

\item К параболе проведены две касательные $l_a$ и $l_b$ в точках $A$ и $B$. Точка $C$ симметрична точке $A$ относительно $l_b$, а точка $D$ симметрична точке $B$ относительно $l_a$. Докажите, что точки $A$, $B$, $C$ и $D$ образуют ромб тогда и только тогда, когда $AB$ проходит через фокус параболы.

\textit{Можно использовать оптическое свойство параболы без доказательства.}

\item Найдите все такие натуральные $n$, при которых 
$$C_{n}^0 \cdot C_{n}^1 \cdot \hdots \cdot C_n^n$$
является точным квадратом.

\textit{Здесь $C_n^k = \dfrac{n!}{k!(n-k)!}$ --- биномиальный коэффициент.}

\item Докажите сходимость последовательности $\{a_n\}$, где 
$$
\begin{cases}
a_1 = 1,\\
a_{n+1} = a_n + \sin{a_n},\\
\end{cases}
$$
и найдите её предел.

\item Функция $F(n, k)$, определённая для всех целых неотрицательных $n$ и $k$, удовлетворяет условиям
$$
\begin{cases}
F(n, k) = F(n - 1, k) + F(n, k - 1), & n \geqslant 1, k \geqslant 1,\\
F(n, 0) = 1,  & n \geqslant 0,\\
F(0, k) = 2 F(0, k - 1) + 1, & k \geqslant 1.
\end{cases}
$$
Вычислите $F(n, n)$.
\end{enumerate}

\end{document} 
